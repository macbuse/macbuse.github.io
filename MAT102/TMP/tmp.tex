\documentclass[a4paper,12pt]{article}
\usepackage{amsmath, amssymb}

\begin{document}

\section*{Sommes géométriques complexes}

\subsection*{Formule générale}

Pour tout \( q \neq 1 \) :
\[
\sum_{n=0}^{N} q^n = \frac{1 - q^{N+1}}{1 - q}
\]

---

\subsection*{(d)}
\[
\sum_{l=0}^{7} (-1 + i)^l
\]

Ici :
\[
q = -1 + i, \quad N = 7
\]

Calculons \( q^8 \) :

\[
(-1 + i) = \sqrt{2}\, e^{i3\pi/4}
\]
\[
q^8 = (\sqrt{2})^8 e^{i8(3\pi/4)} = 2^4 e^{i6\pi} = 16
\]

Ainsi :
\[
S_d = \frac{1 - q^{8}}{1 - q} = \frac{1 - 16}{1 - (-1 + i)} = \frac{-15}{2 - i}
\]

Multiplions par le conjugué \(2 + i\) :
\[
S_d = \frac{-15(2 + i)}{(2 - i)(2 + i)} = \frac{-15(2 + i)}{5} = -3(2 + i) = -6 - 3i
\]

\[
\boxed{S_d = -6 - 3i}
\]

---

\subsection*{(e)}
\[
\sum_{k=0}^{7} (1 + i)^k
\]

Ici :
\[
q = 1 + i, \quad N = 7
\]

\[
(1 + i) = \sqrt{2}\, e^{i\pi/4} \quad \Rightarrow \quad q^8 = (\sqrt{2})^8 e^{i8(\pi/4)} = 2^4 e^{i2\pi} = 16
\]

Ainsi :
\[
S_e = \frac{1 - q^8}{1 - q} = \frac{1 - 16}{1 - (1 + i)} = \frac{-15}{-i} = -15i
\]

\[
\boxed{S_e = -15i}
\]

---

\subsection*{(f)}
\[
\sum_{m=0}^{12} e^{i(2\pi m / 3)}
\]

Ici :
\[
q = e^{i(2\pi/3)}
\]

C’est une racine cubique de l’unité :
\[
1 + q + q^2 = 0, \quad \text{et} \quad q^3 = 1
\]

Les termes se répètent tous les trois :
\[
(1, q, q^2, 1, q, q^2, \dots)
\]

Puisqu’il y a 12 termes, soit 4 cycles complets :
\[
\sum_{m=0}^{12} q^m = 4(1 + q + q^2) = 4 \times 0 = 0
\]

\[
\boxed{S_f = 0}
\]

---

\subsection*{Résultats finaux}

\[
\boxed{
\begin{aligned}
d) & = -6 - 3i \\
e) & = -15i \\
f) & = 0
\end{aligned}
}
\]

\end{document}

