\documentclass[12pt,fleqn]{article}
\usepackage{graphicx}
\usepackage[T1]{fontenc}
\usepackage[utf8]{inputenc}
\usepackage[francais]{babel}
\usepackage{amsmath}
\usepackage{amssymb}
\usepackage{amscd}
\usepackage{latexsym}
\usepackage{tabularx}
\usepackage{a4wide}
%\usepackage{here}
\usepackage{enumerate}
%\usepackage{subfigure}
%\usepackage{stmaryrd}
%\usepackage{vmargin}
\usepackage{url}
\usepackage{marvosym} % Pour les euros
%\usepackage[textwidth=16cm,textheight=23cm,includefoot]{geometry}


%\renewcommand{\baselinestretch}{1.05}\small\normalsize


\newtheorem{theorem}{Theorem}
\newtheorem{conjecture}[theorem]{Conjecture}
\newtheorem{question}{Question}


\begin{document}

\begin{center}
{\huge \bf Proposal for Special Session
for the 2021 AMS-SMF-EMS Joint International Mathematics Meeting in Grenoble
}
\end{center}
\smallskip

\section{Organizers}
It is required that at least one of the organizers be affiliated with a US institution
and at least another one be affiliated with a French institution.

\paragraph{Contact:} name, affiliation, and e-mail address

\paragraph{Second organizer:} name, affiliation, and e-mail address

\paragraph{Third organizer:} name, affiliation, and e-mail address


\section{Session description}

\paragraph{Title:} {\large Combinatorial and computational aspects in Topology}

Our session will be composed of 12 talks: 
3 long/plenary talks 12 short exposés for a total of 9h.

\subsection{Context}

The mathematical focus of the session includes all aspects of the topology and geometry of low-dimensional manifolds and some geometric group theory. It has been understood for over a century that these subjects are tightly connected, but the connections have become even deeper as the subjects have matured. Recent advances have given dramatic evidence of this. The session aims to provide a forum  young and senior researchers from the USA and Europe an occasion to discuss these exciting recent results.

Algorithms have been an important and consistent feature of all of these mathematical areas from the beginning. This includes both questions about the existence of algorithms and the development of practical algorithms for computing natural invariants. More recently, computer experiments and rigorous computer-assisted proofs have had a significant impact. It is natural to expect experimental and computational methods to play an expanding role in the theory of low dimensional spaces. Additional goals of the session are to present the development of new computational tools and implementations of new algorithms, and to provide opportunities for researchers to become more familiar with existing tools and how they can be applied in research.


\subsection{Important progress}

The longer talks (see table in Section 3) will be on the the following topics which we consider to be of particular importance:

\begin{itemize}


\item Lackenby's proof of polynomial bounds for the number of Reidemeister moves for unknot diagrams \cite{la1} (and extensions to knot isotopy problems \cite{la2}).


\item Work of Burton et al. on the census of manifold \cite{bu1,bu2}. 


\item  Work of Wagner et al. \cite{wa1,wa2} on the complexity of the Embedding problem: Deciding if a simplicial complex embeds into $\mathbb{R}^d$ or, for that matter, some other simplicial complex. 


\end{itemize}

\section{Speakers suggestions}
%3. a list of speakers (along with their institutions) whom the organizers plan to
%invite. (It is not necessary to have received confirmed commitments from these
%potential speakers.)

In view of the prevailing uncertainty due to the COVID crisis we have tried to avoid speakers located in America or Australia.


\begin{center}
\begin{tabular}[h]{|c|c|c|l|}
  \hline
 Name & Institution & Talk type \\
\hline \hline
Marc Lackenby & University of Oxford & Long \\
  \hline
Sergei  Matveev & Chelyabinsk and  Novosibirsk State Universities, Russia& Long \\
  \hline
Uli Wagner & Institut of Science and Technology (IST), Austria & Long \\
  \hline
  \hline
 Mark Bell &  University of Warwick & Short \\
  \hline
 Sergio Cabello &  University of Ljubljana & Short \\
  \hline
 Ivan Dynnikov &  St. Petersburg State University & Short \\
  \hline
 Radoslav  Fulek &  University of Arizona & Short \\
  \hline
 Elise Goujard &  Université de Bordeaux & Short \\
  \hline
 Kristóf Huszár &  Institut of Science and Technology (IST), Austria & Short \\
  \hline
 Luke Jeffreys &  University of Glasgow & Short \\
  \hline
 Clément Maria &  INRIA, Sophia Antipolis & Short \\
  \hline
 Arnaud de Mesmay & CNRS, Université de Marne-la-Vallée & Short \\
  \hline
 Aleksandra Skripchenko &  Skoltech Faculty, Moscow & Short \\
  \hline
 Martin Tancer &  Charles University, Prague & Short \\
  \hline
 Mehdi Yazdi &  University of Oxford & Short \\
  \hline
\end{tabular}
\end{center}


\begin{thebibliography}{1}

\bibitem{la1} 
A polynomial upper bound on Reidemeister moves Annals Math. 182 (2015) 491-564
\bibitem{la2} 
The computational complexity of determining knot genus in a fixed 3-manifold.
\url{ http://people.maths.ox.ac.uk/lackenby/KnotGenusComplexity2April2020.pdf}
\bibitem{bu1}
Benjamin A. Burton, Sergio Cabello, Stefan Kratsch, William Pettersson:
The Parameterized Complexity of Finding a 2-Sphere in a Simplicial Complex. SIAM J. Discret. Math. 33(4): 2092-2110 (2019)
\bibitem{bu2}
Benjamin A. Burton, Alexander He:
On the hardness of finding normal surfaces. \url{https://arxiv.org/abs/1912.09051}
\bibitem{wa1}
M. Filakovský, U. Wagner, S.Y. Zhechev: Embeddability of simplicial complexes is undecidable,
Proceedings of the Annual ACM-SIAM Symposium on Discrete Algorithms, SIAM, 2020, pp. 767–785. 
\bibitem{wa2}
Matoušek J, Sedgwick E, Tancer M, Wagner U. Embeddability in the 3-Sphere is decidable. Journal of the ACM. 2018;65(1):5. doi:10.1145/3078632

\end{thebibliography}



\end{document}