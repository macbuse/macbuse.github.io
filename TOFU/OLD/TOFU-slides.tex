% Classe, package, etc. 
% L'option ``compress'' dans le documentclass permet d'avoir une table
% des matières compressée en haut des transparents
\documentclass[a4paper,compress]{beamer}  %  [handout]{beamer}

%\mode<presentation>{\usetheme{Hannover}}
\usepackage[francais]{babel}
\usepackage[utf8]{inputenc}
\usepackage[T1]{fontenc}
% \usepackage{lmodern}
%\usepackage[english]{babel}
% \usepackage[babel]{microtype}
\usepackage{xcolor,colortbl}
\usepackage{graphicx,url}
\usepackage{enumerate}
\usepackage{amssymb,amsthm}
\usepackage{times}
%\usepackage{amsmath,amscd}
\usepackage[all]{xy}
%\usepackage{soul} % pour barrer du texte
\usepackage{url}\urlstyle{same}
\usepackage{marvosym} % Pour les euros

% Automatic import of SVG, replacing text with LaTeX
% Compile with: (pdf)latex -shell-escape ...
\newcommand{\executeiffilenewer}[3]{%
\ifnum\pdfstrcmp{\pdffilemoddate{#1}}%
{\pdffilemoddate{#2}}>0%
{\immediate\write18{#3}}\fi%
}
\def\svgmode{ps}\ifx\pdfoutput\undefined
\def\executeiffilenewer#1#2#3{\immediate\write18{#3}}\else%\bibliography{bibli}

\ifnum\pdfoutput<1\else\def\svgmode{pdf}\fi\fi
\newcommand{\includesvgX}[1]{%
\executeiffilenewer{#1.svg}{#1.\svgmode}%
{inkscape -z -D --file=#1.svg --export-latex %
--export-\svgmode=#1.\svgmode}%
\input{#1.\svgmode_tex}%
}
\newcommand{\includesvg}[2][]{%
\def\tempa{#1}\def\tempb{}%
\ifx\tempa\tempb\else\let\svgwidth\tempa\fi
\includesvgX{\svgpath#2}%
}



%%%%%%%%%%%%%%%%%%%%%%%%%%%%%%%%%%%%%%%%%%%%%%%%%%%%%%%%%%%%%%%%%%%%%%%%%%
% A few shortcuts..
%\def\scal#1{\hbox{\rm#1}}
\def\vect#1{\hbox{\bf#1}}
\def\Real{{\rm I\hskip-2pt R}}
\def\Integer{{\rm Z\hskip-4pt Z}}
\def\sus#1{{\hbox{\scriptsize #1}}}

\def\derivOne#1#2{{\scal d#1 \over \scal d#2}}
\def\derivTwo#1#2{{\scal d^2#1 \over \scal d#2^2}}
\def\partialOne#1#2{{\partial#1 \over \partial#2}}
\def\partialTwoSame#1#2{{\partial^2#1 \over \partial#2^2}}
\def\partialTwo#1#2#3{{\partial^2#1 \over \partial#2\,\partial#3}}
\def\partialThree#1#2#3#4{{\partial^3#1
 \over \partial#2\, \partial#3\, \partial#4}}
\def\partialThree_TwoOne#1#2#3{{\partial^3#1 \over \partial#2^2\, \partial#3}}
\def\partialThree_OneTwo#1#2#3{{\partial^3#1 \over \partial #2\, \partial#3^2}}
\def\partialThreeSame#1#2{{\partial^3#1 \over \partial#2^3}}

\def\inprod#1#2{\langle#1, #2\rangle}
% ---- DELIMITER PAIRS ---- from Jeff Erickson
\def\floor#1{\lfloor #1 \rfloor}
\def\ceil#1{\lceil #1 \rceil}
\def\seq#1{\langle #1 \rangle}
\def\set#1{\{ #1 \}}
\def\abs#1{\mathopen| #1 \mathclose|}	% use instead of $|x|$ 
\def\norm#1{\mathopen\| #1 \mathclose\|}	% use instead of $\|x\|$ 
\def\indic#1{\big[#1\big]}                              % indicator variable; Iverson notation
					                                % e.g., Kronecker delta = [x=0]
\def\dual#1{{#1}^{\Delta}}
\def\bidual#1{{#1}^{\Delta\Delta}}

\newcommand{\ematrix}{\end{array}\right)}
\newcommand{\Alpha}{\mbox{\bf \LARGE $\alpha$}}
\newcommand{\Kappa}{\mbox{\bf \LARGE $\kappa$}}
\newcommand{\degr}{$\!\!^\circ$}
% \newcommand{\Z}{\mbox{\rm Z\hskip-4pt Z}}
\newcommand{\Z}{\mathbb{Z}}
%\newcommand{\N}{\mbox{\rm I\hspace{-.2em}N}}
\newcommand{\N}{\mathbb{N}}
\newcommand{\Reel}{\mbox{\rm I\hspace{-.2em}R}}
                %%  ATTENTION ne pas retirer les espaces dans \Complex !!!
\newcommand{\R}{\mathbb{R}}
% \newcommand{\R}{\mbox{\rm I\hspace{-.2em}R}}
                %%  ATTENTION ne pas retirer les espaces dans \Complex !!!
\newcommand{\C}{\mathbb{C}}
%\newcommand{\C}{\mbox{ \rm \rule[.05ex]{.15ex}{1.4ex}\hspace{-1.4ex} C}}
\newcommand{\Complex}{\mbox{ \rm \rule[.05ex]{.15ex}{1.4ex}\hspace{-1.4ex} C}}
\newcommand{\Q}{\mathbb{Q}}
\providecommand{\1}{\mbox{\rm 1\hspace{-.27em}I}}
\newcommand{\0}{\mbox{\rm I\hspace{-.45em}}\mathbf{0}}
\newcommand{\zbf}{\mathbf{0}}

%% \newcommand{\binom}[2]{\left(\!\!\!\begin{array}{c} #1 \\
%% #2\end{array}\!\!\!\right)}
        %% binom d�fini dans amsmath
\newcommand{\matundeux}[2]{\left[\!\!\!\begin{array}{c} #1 \\
     #2\end{array}\!\!\!\right]}

\newcommand{\matuntrois}[3]{\left[\!\!\!\begin{array}{c} #1 \\
     #2 \\ #3\end{array}\!\!\!\right]}
%% \newcommand{\implies}{\Rightarrow} 
        %% implies d�fini dans amsmath
\newcommand{\isequ}{\Leftrightarrow}
\newcommand{\inter}[1]{\stackrel{\circ}{#1}}
\newcommand{\clos}[1]{\bar{#1}}
\newcommand{\complt}[1]{\complement{#1}}
\newcommand{\contract}{\mbox{/\hspace{-.2em}/}}
\newcommand{\inv}{\ensuremath{^{-1}}}
\newcommand{\onto}{\twoheadrightarrow}
\newcommand{\into}{\hookrightarrow}

\makeatletter\def\proof{\@ifnextchar[{\@xproof}{\@proof}}
\def\endproof{\unskip\kern 10\p@ \begingroup \unitlength\p@
  \linethickness{.4\p@} \framebox(6,6){} \endgroup \endtrivlist }
\def\@proof{\trivlist \item[ \hskip 10\p@ \hskip \labelsep {\sc
    Proof.}  ] \ignorespaces }
\def\@xproof[#1]{\trivlist \item[\hskip 10\p@\hskip \labelsep{\sc
    Proof #1.}] \ignorespaces } \makeatother

\makeatletter\def\skproof{\@ifnextchar[{\@xskproof}{\@skproof}}
\def\endskproof{\unskip\kern 10\p@ \begingroup \unitlength\p@
  \linethickness{.4\p@} \framebox(6,6){} \endgroup \endtrivlist }
\def\@skproof{\trivlist \item[ \hskip 10\p@ \hskip \labelsep {\sc
    Sketch of Proof.}  ] \ignorespaces }
\def\@skxproof[#1]{\trivlist \item[\hskip 10\p@\hskip \labelsep{\sc
    Sketch of Proof #1.}] \ignorespaces } \makeatother

\makeatletter\def\preuve{\@ifnextchar[{\@xpreuve}{\@preuve}}
\def\endpreuve{\unskip\kern 10\p@ \begingroup \unitlength\p@
  \linethickness{.4\p@} \hfill $\Box${} \endgroup \endtrivlist }
\def\@preuve{\trivlist \item[  {\bf
    Preuve :}  ] \ignorespaces }
\def\@xpreuve[#1]{\trivlist \item[\hskip \labelsep{\bf
    Preuve #1 :}] \ignorespaces } \makeatother

\makeatletter\def\skpreuve{\@ifnextchar[{\@xskpreuve}{\@skpreuve}}
\def\endskpreuve{\unskip\kern 10\p@ \begingroup \unitlength\p@
  \linethickness{.4\p@} \framebox(6,6){} \endgroup \endtrivlist }
\def\@skpreuve{\trivlist \item[ \hskip 10\p@ \hskip \labelsep {\sc
    Sketch of Preuve.}  ] \ignorespaces }
\def\@skxpreuve[#1]{\trivlist \item[\hskip 10\p@\hskip \labelsep{\sc
    Sketch of Preuve #1.}] \ignorespaces } \makeatother
%%% Local Variables: 
%%% mode: plain-tex
%%% TeX-master: "math_macros"
%%% End: 

\def\scal#1{\hbox{\rm#1}}
\def\vect#1{\hbox{\bf#1}}
\def\Real{{\rm I\hskip-2pt R}}
\def\Integer{{\rm Z\hskip-4pt Z}}
\def\sus#1{{\hbox{\scriptsize #1}}}

\def\derivOne#1#2{{\scal d#1 \over \scal d#2}}
\def\derivTwo#1#2{{\scal d^2#1 \over \scal d#2^2}}
\def\partialOne#1#2{{\partial#1 \over \partial#2}}
\def\partialTwoSame#1#2{{\partial^2#1 \over \partial#2^2}}
\def\partialTwo#1#2#3{{\partial^2#1 \over \partial#2\,\partial#3}}
\def\partialThree#1#2#3#4{{\partial^3#1
 \over \partial#2\, \partial#3\, \partial#4}}
\def\partialThree_TwoOne#1#2#3{{\partial^3#1 \over \partial#2^2\, \partial#3}}
\def\partialThree_OneTwo#1#2#3{{\partial^3#1 \over \partial #2\, \partial#3^2}}
\def\partialThreeSame#1#2{{\partial^3#1 \over \partial#2^3}}

\def\inprod#1#2{\langle#1, #2\rangle}
% ---- DELIMITER PAIRS ---- from Jeff Erickson
\def\floor#1{\lfloor #1 \rfloor}
\def\ceil#1{\lceil #1 \rceil}
\def\seq#1{\langle #1 \rangle}
\def\set#1{\{ #1 \}}
\def\abs#1{\mathopen| #1 \mathclose|}	% use instead of $|x|$ 
\def\norm#1{\mathopen\| #1 \mathclose\|}	% use instead of $\|x\|$ 
\def\indic#1{\big[#1\big]}                              % indicator variable; Iverson notation
					                                % e.g., Kronecker delta = [x=0]
\def\dual#1{{#1}^{\Delta}}
\def\bidual#1{{#1}^{\Delta\Delta}}

\newcommand{\ematrix}{\end{array}\right)}
\newcommand{\Alpha}{\mbox{\bf \LARGE $\alpha$}}
\newcommand{\Kappa}{\mbox{\bf \LARGE $\kappa$}}
\newcommand{\degr}{$\!\!^\circ$}
% \newcommand{\Z}{\mbox{\rm Z\hskip-4pt Z}}
\newcommand{\Z}{\mathbb{Z}}
%\newcommand{\N}{\mbox{\rm I\hspace{-.2em}N}}
\newcommand{\N}{\mathbb{N}}
\newcommand{\Reel}{\mbox{\rm I\hspace{-.2em}R}}
                %%  ATTENTION ne pas retirer les espaces dans \Complex !!!
\newcommand{\R}{\mathbb{R}}
% \newcommand{\R}{\mbox{\rm I\hspace{-.2em}R}}
                %%  ATTENTION ne pas retirer les espaces dans \Complex !!!
\newcommand{\C}{\mathbb{C}}
%\newcommand{\C}{\mbox{ \rm \rule[.05ex]{.15ex}{1.4ex}\hspace{-1.4ex} C}}
\newcommand{\Complex}{\mbox{ \rm \rule[.05ex]{.15ex}{1.4ex}\hspace{-1.4ex} C}}
\newcommand{\Q}{\mathbb{Q}}
\providecommand{\1}{\mbox{\rm 1\hspace{-.27em}I}}
\newcommand{\0}{\mbox{\rm I\hspace{-.45em}}\mathbf{0}}
\newcommand{\zbf}{\mathbf{0}}

%% \newcommand{\binom}[2]{\left(\!\!\!\begin{array}{c} #1 \\
%% #2\end{array}\!\!\!\right)}
        %% binom d�fini dans amsmath
\newcommand{\matundeux}[2]{\left[\!\!\!\begin{array}{c} #1 \\
     #2\end{array}\!\!\!\right]}

\newcommand{\matuntrois}[3]{\left[\!\!\!\begin{array}{c} #1 \\
     #2 \\ #3\end{array}\!\!\!\right]}
%% \newcommand{\implies}{\Rightarrow} 
        %% implies d�fini dans amsmath
\newcommand{\isequ}{\Leftrightarrow}
\newcommand{\inter}[1]{\stackrel{\circ}{#1}}
\newcommand{\clos}[1]{\bar{#1}}
\newcommand{\complt}[1]{\complement{#1}}
\newcommand{\contract}{\mbox{/\hspace{-.2em}/}}
\newcommand{\inv}{\ensuremath{^{-1}}}
\newcommand{\onto}{\twoheadrightarrow}
\newcommand{\into}{\hookrightarrow}

\makeatletter\def\proof{\@ifnextchar[{\@xproof}{\@proof}}
\def\endproof{\unskip\kern 10\p@ \begingroup \unitlength\p@
  \linethickness{.4\p@} \framebox(6,6){} \endgroup \endtrivlist }
\def\@proof{\trivlist \item[ \hskip 10\p@ \hskip \labelsep {\sc
    Proof.}  ] \ignorespaces }
\def\@xproof[#1]{\trivlist \item[\hskip 10\p@\hskip \labelsep{\sc
    Proof #1.}] \ignorespaces } \makeatother

\makeatletter\def\skproof{\@ifnextchar[{\@xskproof}{\@skproof}}
\def\endskproof{\unskip\kern 10\p@ \begingroup \unitlength\p@
  \linethickness{.4\p@} \framebox(6,6){} \endgroup \endtrivlist }
\def\@skproof{\trivlist \item[ \hskip 10\p@ \hskip \labelsep {\sc
    Sketch of Proof.}  ] \ignorespaces }
\def\@skxproof[#1]{\trivlist \item[\hskip 10\p@\hskip \labelsep{\sc
    Sketch of Proof #1.}] \ignorespaces } \makeatother

\makeatletter\def\preuve{\@ifnextchar[{\@xpreuve}{\@preuve}}
\def\endpreuve{\unskip\kern 10\p@ \begingroup \unitlength\p@
  \linethickness{.4\p@} \hfill $\Box${} \endgroup \endtrivlist }
\def\@preuve{\trivlist \item[  {\bf
    Preuve :}  ] \ignorespaces }
\def\@xpreuve[#1]{\trivlist \item[\hskip \labelsep{\bf
    Preuve #1 :}] \ignorespaces } \makeatother

\makeatletter\def\skpreuve{\@ifnextchar[{\@xskpreuve}{\@skpreuve}}
\def\endskpreuve{\unskip\kern 10\p@ \begingroup \unitlength\p@
  \linethickness{.4\p@} \framebox(6,6){} \endgroup \endtrivlist }
\def\@skpreuve{\trivlist \item[ \hskip 10\p@ \hskip \labelsep {\sc
    Sketch of Preuve.}  ] \ignorespaces }
\def\@skxpreuve[#1]{\trivlist \item[\hskip 10\p@\hskip \labelsep{\sc
    Sketch of Preuve #1.}] \ignorespaces } \makeatother
%%% Local Variables: 
%%% mode: plain-tex
%%% TeX-master: "math_macros"
%%% End: 


\def\deriv#1#2{{#1 \!\!\cdot\!\! #2}}


%%%%%%%%%%%%%%%%%%%%%%%%%%%%%%%%%%%%%%%%%%%%%%%%%%%%%%%%%%%%%%%%%%%%%%%%%%
% graphics
\graphicspath{{figures/}{../../PRESENTATION/GRATOS/figures/}}
\def\svgpath{figures/}

%\DeclareGraphicsExtensions{.eps,.eps.gz}
%\DeclareGraphicsRule{.eps.gz}{eps}{.eps.bb}{`gunzip -c #1}

%-------------------------------------------------------------------------
% Thème des transparents
\usetheme{Warsaw} %{default}
%\usetheme{classic}
\setbeamercovered{invisible}

% pour le problème d'affichage du texte après les blocks
% \let\origframetitle\frametitle
% \renewcommand{\frametitle}[1]{%
%   \origframetitle{#1}\usebeamercolor*{normal text}%
% }
%---------------------------------------------------------------------------
% Il existe aussi des thèmes de couleurs
%\usecolortheme{crane}

%--------------------------------------------------------------------------
% Pour que seule la section courante apparaisse dans le bandeau de
% table des matières
%\setbeamertemplate{section in head/foot shaded}{}
%\setbeamertemplate{subsection in head/foot shaded}{}
\setbeamertemplate{footline}{}

%--------------------------------------------------------------------------
% Niveau de transparence des éléments grisés avant d'être affichés 
\beamertemplatetransparentcoveredmedium

%-------------------------------------------------------------------------
% Définitions de couleurs
\definecolor{vert}{rgb}{0,255,0}
\definecolor{darkgreen}{rgb}{0,0.3,0}
\definecolor{rouge}{rgb}{0.75,0,0}
\definecolor{lightblue}{rgb}{0,255,255}
\definecolor{myorange}{RGB}{255,165,0}
\definecolor{mauve}{rgb}{201,0,224}
\definecolor{kaki}{rgb}{0.13,0.47,0.4}  % {33,120,103}
\definecolor{LightCyan}{rgb}{0.88,1,1}
\definecolor{LightGreen1}{rgb}{0.7,1,0.5} %{179,255,128}
\definecolor{LightGreen2}{rgb}{.5,1,.16} %{127,255,42}
\definecolor{LightGreen3}{rgb}{.33,.83,0} %{85,212,0}
\definecolor{LightGreen4}{rgb}{.26,.66,0} %{68,170,0}
\definecolor{LightBlue1}{RGB}{170,255,238}
\definecolor{LightBlue2}{HTML}{77FFDD}
\definecolor{LightBlue3}{RGB}{0,255,204}
\definecolor{LightOrange2}{HTML}{FFB380}
\definecolor{LightOrange1}{HTML}{FFCCAA}

\newcommand{\col}{\textcolor}
\newcommand{\red}{\textcolor{red}}
\newcommand{\green}{\textcolor{green}}
\newcommand{\kaki}{\textcolor{kaki}}
\newcommand{\darkgreen}{\textcolor{darkgreen}}
\newcommand{\purple}{\textcolor{mauve}}
\newcommand{\lightblue}{\textcolor{lightblue}}
\newcommand{\blue}{\textcolor{blue}}

%--------------------------------------------------------------------------
\newcommand{\E}{\mathbb{E}}
\newcommand{\Sph}{\mathbb{S}}
\newcommand{\torus}{{\mathbb{T}^2}}
\newcommand{\scalprod}[3]{{\langle #1, #2 \rangle}_{\E^{#3}}}
\newcommand{\matdeuxdeux}[4]{\left(\!\!\!\begin{array}{cc} #1 & #2\\
     #3 & #4\end{array}\!\!\!\right)}

%--------------------------------------------------------------------------
\theoremstyle{definition}
\newtheorem{theo}{Théorème}
\newtheorem{prop}[theo]{Proposition}
\newtheorem{defi}[theo]{Définition}
\newtheorem{lemme}[theo]{Lemme}
\newtheorem{cor}[theo]{Corollaire}
\newtheorem{exmpl}[theo]{Example}

\newenvironment{mydefi}[1][Définition]{\begin{block}{#1}}{\end{block}} 
\newenvironment{mytheo}[1][Théorème]{\begin{alertblock}{#1}}{\end{alertblock}} 
\newenvironment{myprop}[1][Proposition]{\begin{alertblock}{#1}}{\end{alertblock}} 
\newenvironment{mycor}[1][Corollaire]{\begin{alertblock}{#1}}{\end{alertblock}} 
\newenvironment{mylem}[1][Lemme]{\begin{exampleblock}{#1}}{\end{exampleblock}} 
\newenvironment{myexmpl}[1][Exemple]{\begin{exampleblock}{#1}}{\end{exampleblock}} 

%--------------------------------------------------------------------------
% Pour mettre les numéros des transparents en bas de page
%\setbeamertemplate{footline}
% {\begin{center}
% \insertframenumber
% \end{center}}
 
%-------------------------------------------------------------------------
% Gestion des symboles de navigation : cette ligne permet de les
% supprimer (ils y sont par défaut)
\setbeamertemplate{navigation symbols}{}


%\setbeamercolor{normal text}{bg=RGB={0,0,150}}
%\usecolortheme{albatross}
%--------------------------------------------------------------------------
%Page de titre
\title[PERSYVAL-Lab]{Projet équipe-action\\
{\bf Méthodes géométriques en combinatoire,\\
algorithmes combinatoires en géométrie
}}


%\institute{GIPSA-Lab, CNRS, Grenoble}

\date{}

\setbeamertemplate{headline}[default]
\begin{document}

% ----------------  TITRE ----------------------------------------
%\setbeamertemplate{background canvas}[vertical shading]{top=darkblue,bottom=darkblue}

\begin{frame}
\titlepage{}
\end{frame}
% ----------------------------------------------------------------
\begin{frame}
\frametitle{Composition de l'équipe}
%\small
\begin{center}
\textbf{8 membres, 6 équipes, 3 laboratoires}
\vspace{1cm}

\begin{tabular}{|l  | l @{} |l|}
  \hline
 Laboratoire & Équipe & Nom \\
\hline \hline
\rowcolor{LightBlue2}
\cellcolor{LightBlue1}G-SCOP & {\small Optimisation Combinatoire} & {\bf Louis Esperet} \\
\hline
\rowcolor{LightBlue2}
\cellcolor{LightBlue1}G-SCOP & {\small Optimisation Combinatoire} & {\bf Andr\'as Seb\H{o}} \\
\hline
\rowcolor{LightBlue2}
\cellcolor{LightBlue1}G-SCOP & {\small Optimisation Combinatoire}& {\bf Gautier Stauffer} \\
\hline
\rowcolor{LightBlue3}
\cellcolor{LightBlue1}G-SCOP & {\small Recherche Opérationelle}& {\bf Nicolas Catusse} \\
\hline
\rowcolor{LightGreen2}
\cellcolor{LightGreen1}IF & {\small Théorie des Nombres} & {\bf Roland Bacher} \\
\hline
\rowcolor{LightGreen3}
\cellcolor{LightGreen1}IF & {\small Physique Mathématique}& {\bf Yves Colin De Verdière} \\
\hline
\rowcolor{LightGreen4}
\cellcolor{LightGreen1}IF & {\small Topologie} & {\bf François Dahmani} \\
\hline
\rowcolor{blue!25}
\cellcolor{blue!10}GIPSA & {\small AGPIG} & {\bf Francis Lazarus} \\
\hline
\end{tabular}
\end{center}

\end{frame}
%---------------------------------------------------
\begin{frame}
  \frametitle{Les objectifs}

Rassembler des forces vives \alert{éparpillées} sur le site grenoblois
concernant les \alert{aspects discrets de la géométrie/topologie} pour
\vspace{.5cm}
\begin{itemize}
\item Mieux \alert{maîtriser les modèles} inhérents aux 
  simulations informatiques du monde réel. 
\item S'attaquer à quelques \alert{conjectures célèbres} du domaine.
\item Apporter de \alert{nouveaux outils} pour l'analyse de problèmes
  combinatoires de nature topologique.
\end{itemize}
\vspace{.5cm}
Un credo : le site grenoblois possède un fort potentiel inexploité
\emph{qui le restera sans une
  politique volontariste des tutelles}.
\end{frame}
%---------------------------------------------------
\begin{frame}
  \frametitle{Trois thématiques}
\Large
  \begin{itemize}
  \item Graphes Plongés 
\item Topologie de dimension supérieure
\item Graphes et géométrie
  \end{itemize}
\end{frame}
%---------------------------------------------------
\begin{frame}
  \frametitle{Graphes Plongés}
\centerline{
\includegraphics<1>[width=.8\linewidth]{combiSurf1.pdf}
\includegraphics<2>[width=.8\linewidth]{eight.pdf}
}
\end{frame}
%---------------------------------------------------
\begin{frame}
  \frametitle{Cycle de partage}
\begin{center}
  \begin{exampleblock}{Conjecture (Barnette, 1982)}
    Toute triangulation de $S_{g>1}$ admet un cycle de partage.
  \end{exampleblock}
\vspace{.5cm}
   \includegraphics[width=\linewidth]{splitting-fr}
\end{center}
\visible<2>{
Il suffit de vérifier la conjecture pour les triangulations irréductibles. Pour $g=2$ il y en a $396\,784$. }
\end{frame}
%---------------------------------------------------
\begin{frame}
  \frametitle{Design Topologique}
Un 0-système possède au plus $3g-3$ courbes.
\vspace{1cm}

\centerline{
\includegraphics<1>[width=.8\linewidth]{pantsDecomposition-0.pdf}
\includegraphics<2>[width=.8\linewidth]{pantsDecomposition-1.pdf}
\includegraphics<3>[width=.8\linewidth]{pantsDecomposition-2.pdf}
}
\end{frame}
%---------------------------------------------------
\begin{frame}
  \frametitle{Design Topologique}
Un 1-système possède au plus $N(1,g)$ courbes avec
\[ g^2 + \frac{5}{2}g \leq N(1,g) \leq (g-1)(2^{2g} - 1)
\]
\centerline{
\includegraphics[width=.8\linewidth]{homol-curves.pdf}
}
\end{frame}
%---------------------------------------------------
\begin{frame}
  \frametitle{Genre des colorations de Tait}
\centerline{
\includegraphics<1>[width=1.2\linewidth]{tait-coloring-1.pdf}
\includegraphics<2>[width=1.2\linewidth]{tait-coloring-2.pdf}
}
\end{frame}
%---------------------------------------------------
\begin{frame}
  \frametitle{Bibliothèque dédiée aux courbes sur les surfaces}
\centerline{
\includegraphics[width=.4\linewidth]{graphics-1.png}
\includegraphics[width=.4\linewidth]{graphics-2.png}
\includegraphics[width=.4\linewidth]{graphics-3.png}
}
\vspace*{1cm}

Applications en CAO, modélisation pour la synthèse d'images...
\end{frame}
%---------------------------------------------------
\begin{frame}
  \frametitle{Topologie en dimension supérieure}
\only<1>
{\bf Algorithmique de la théorie des n\oe uds}
\centerline{
\includegraphics<1>[width=.6\linewidth]{knot_g9-wb.jpg}%
\includegraphics<2>[width=\linewidth]{normal-surf.pdf}%
\includegraphics<3>[width=\linewidth]{matching-conds-0.pdf}%
\includegraphics<4->[width=\linewidth]{matching-conds-1.pdf}%
}
\only<2-4>{
\only<2>{\vspace*{1.4cm}}
\begin{block}{}
  Une surface \alert{normale} dans une triangulation à $t$ tétraèdres se code par 
un $v\in \N^{7t}$. \visible<4>{$v$ doit vérifier des \alert{conditions d'appariements} (linéaires) et \alert{conditions quadrilatèrales} de non-croisements (non-linéaires).}
\end{block}
}
\only<5>{
\begin{alertblock}{}
$v$ appartient au \alert{cône normal de Haken}. On peut se restreindre aux éléments d'une \alert{base hilbertienne} de ce cône.
\end{alertblock}
}
\end{frame}
%---------------------------------------------------
\begin{frame}
  \frametitle{Graphes et géométrie}
\textbf{Spanners géométriques}\\
Optimiser le facteur d'étirement, le degré maximum, le nombre d'arêtes, la planarité...
\begin{center}
\centerline{
\includegraphics<1>[width=.75\linewidth]{300pt_2.jpg}%
\includegraphics<2>[width=.75\linewidth]{proteinspanner.pdf}%
\includegraphics<3>[width=.75\linewidth]{voronoi.pdf}%
}
\only<1>{Problème de routage avec centres émetteurs et clients.}
\only<2>{Squelette de protéine avec ``court-circuits'' additionnels.}
\only<3>{Problème : trouver le facteur d'étirement de Delaunay.}
\end{center}
\end{frame}
%---------------------------------------------------
\begin{frame}
  \frametitle{Conjecture du coureur solitaire}
\centerline{
\includegraphics[width=.75\linewidth]{runner.jpg}%
}
\end{frame}
%---------------------------------------------------
\begin{frame}
  \frametitle{Activités}
  \begin{itemize}
  \item Groupe(s) de lecture (Matveev, Koszlov, Farb et Margalit,...)
\item Invitations pour séminaires et cours doctoraux
\item colloques
\item ...
  \end{itemize}
\end{frame}
%---------------------------------------------------
\begin{frame}
  \frametitle{Diffusion}
\centerline{\framebox{\includegraphics[height=.85\textheight]{GT-GeoAlg.png}}
}
\end{frame}
%---------------------------------------------------
\begin{frame}
  \frametitle{Demande de moyens}
\begin{itemize}
\item[] {\bf Financement de thèse} : \hfill 100.000 \EUR\\
\item[] {\bf Post-doc} (1,5 an) \hfill 75.000 \EUR\\
\item[] {\bf Gratifications de stage} : \hfill 12.000 \EUR\\
{\footnotesize 6 stages M2 de 5 mois + stages ENS et Polytechnique non-rémunérés (niveau L3,M1,M2)}
\item[] {\bf Invitations de chercheurs extérieurs} : \hfill 40.000 \EUR\\
{\footnotesize Benjamin Burton, Jens Vygen, Gianpaolo Oriolo, Shalom Eliahou, Bill Cook,...
}
\item[] {\bf Missions} : \hfill 30.000 \EUR\\
{\footnotesize (1200\EUR par personne et par an)}
\item[] {\bf Matériel} : \hfill 10.000 \EUR
\item[] {\bf Fonctionnement} : \hfill 4.000 \EUR
\item[] {\bf Congrès-colloques} : \hfill 9.000 \EUR
\item[] \rule[-0.1cm]{\linewidth}{0.01cm} 
\item[] {\bf TOTAL demandé : \hfill 280.000 \EUR}
\end{itemize}
\end{frame}
%---------------------------------------------------
\end{document}
%---------------------------------------------------
\begin{frame}
  \frametitle{}
\end{frame}
