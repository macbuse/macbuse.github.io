\documentclass[12pt]{article}
\usepackage[utf8]{inputenc}
\usepackage[T1]{fontenc}
\usepackage[french]{babel}
\usepackage{amsmath, amssymb}
\usepackage{geometry}
\geometry{a4paper, margin=1in}

\title{Définition et Propriétés de la Médiane}
\author{}
\date{}

\begin{document}

\maketitle


\maketitle

\section{Définition Mathématique}
La \textbf{médiane} d'une variable aléatoire réelle $X$ est une valeur $m$ qui sépare la distribution de probabilité en deux parties égales. Elle est définie par les deux conditions suivantes :
\begin{enumerate}
    \item $P(X \le m) \ge \frac{1}{2}$
    \item $P(X \ge m) \ge \frac{1}{2}$
\end{enumerate}

Pour une variable aléatoire continue de fonction de répartition $F$, la médiane est simplement la valeur $m$ telle que :
$$F(m) = \frac{1}{2}$$

\section{Lien avec les exercices}

\subsection{Minimisation de l'erreur absolue (Exercice 6)}
Une propriété fondamentale de la médiane est qu'elle est la valeur $a$ qui minimise la fonction d'écart absolu :
$$g(a) = \mathbb{E}|X - a|$$
Comme demandé dans l'exercice 6, pour une variable discrète uniforme sur $\{x_1, \dots, x_r\}$, cela revient à minimiser la somme des distances aux points.

\subsection{Calcul pour les lois de l'exercice 5}
À partir des fonctions de répartition calculées précédemment, nous pouvons déduire les médianes théoriques :

\begin{itemize}
    \item \textbf{Loi Béta $B(\alpha, 1)$ :} 
    On résout $F(m) = m^\alpha = \frac{1}{2}$, d'où :
    $$m = \left(\frac{1}{2}\right)^{1/\alpha}$$
    
    \item \textbf{Loi de Weibull :} 
    On résout $F(m) = 1 - e^{-\lambda m^\alpha} = \frac{1}{2}$, ce qui donne $e^{-\lambda m^\alpha} = \frac{1}{2}$. En passant au logarithme :
    $$-\lambda m^\alpha = \ln\left(\frac{1}{2}\right) \implies m = \left( \frac{\ln(2)}{\lambda} \right)^{1/\alpha}$$
\end{itemize}

\section{Comparaison avec la moyenne}
Contrairement à l'espérance $\mathbb{E}[X]$, la médiane est robuste aux valeurs extrêmes (outliers). Dans l'exercice 7, bien que $X$ soit symétrique sur $[-1, 3]$ (moyenne = médiane = $1$), la transformation $Y = h(X)$ peut décaler ces deux valeurs l'une par rapport à l'autre selon la forme de $h$.

\section*{Énoncé}
Montrer que si $X$ est une variable aléatoire réelle, les médianes de $X$ minimisent la fonction $a \mapsto \mathbb{E}|X - a|$. On commence par le cas d'une variable discrète uniforme sur $\{x_1, \dots, x_r\}$ avec $x_1 < \dots < x_r$.

\section{Cas discret uniforme}
Soit $X$ une variable uniforme sur $\{x_1, \dots, x_r\}$. La fonction à minimiser est :
$$g(a) = \frac{1}{r} \sum_{i=1}^{r} |x_i - a|$$

\subsection*{Analyse de la dérivée}
La fonction $g$ est continue et dérivable par morceaux (sauf aux points $x_i$). Sa dérivée est :
$$g'(a) = \frac{1}{r} \sum_{i=1}^{r} -\text{sgn}(x_i - a)$$
où $\text{sgn}(u) = 1$ si $u > 0$ et $-1$ si $u < 0$. 

On peut réécrire cela en comptant le nombre de points à gauche et à droite de $a$ :
$$g'(a) = \frac{1}{r} \left[ \text{card}\{i : x_i < a\} - \text{card}\{i : x_i > a\} \right]$$

\subsection*{Condition d'optimalité}
Le minimum est atteint lorsque la dérivée change de signe (ou s'annule) :
\begin{itemize}
    \item Si $r$ est impair ($r=2k+1$), $g'(a)$ change de signe exactement en $a = x_{k+1}$. La médiane unique est $x_{k+1}$.
    \item Si $r$ est pair ($r=2k$), $g'(a) = 0$ pour tout $a \in ]x_k, x_{k+1}[$. Toutes les valeurs de cet intervalle (les médianes) minimisent $g(a)$.
\end{itemize}

\section{Cas général (Variable à densité)}
Soit $f$ la densité de $X$. On veut minimiser $g(a) = \int_{-\infty}^{+\infty} |x - a| f(x) dx$.
Décomposons l'intégrale :
$$g(a) = \int_{-\infty}^{a} (a - x) f(x) dx + \int_{a}^{+\infty} (x - a) f(x) dx$$

En dérivant par rapport à $a$ (en utilisant la règle de Leibniz) :
$$g'(a) = \int_{-\infty}^{a} f(x) dx - \int_{a}^{+\infty} f(x) dx = P(X \le a) - P(X > a)$$

Pour obtenir le minimum, on pose $g'(a) = 0$ :
$$P(X \le a) = P(X > a)$$
Comme $P(X \le a) + P(X > a) = 1$, cela implique :
$$P(X \le a) = \frac{1}{2}$$
Cette condition définit précisément la médiane de la variable $X$.


\end{document}
