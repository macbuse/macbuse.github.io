\documentclass[11pt,a4paper]{article}
\usepackage{ifthen}
\usepackage{amsmath}
\usepackage{amsthm}
\usepackage{amssymb}
\usepackage{latexsym}
%\usepackage{bbm}
\usepackage{graphicx}
\usepackage{pdfsync}
\usepackage{verbatim}
\usepackage{fancyhdr}
\usepackage{url}               
\usepackage[french]{babel}
\usepackage[applemac]{inputenc}
\usepackage{showkeys}
 \usepackage{color}
 \def\boxit#1#2{\vbox{\hrule\hbox{\vrule
\vbox spread#1{\vfil\hbox spread#1{\hfil#2\hfil}\vfil}%
\vrule}\hrule}}

\long\def\emboite#1#2{\setbox4=\vbox{\hsize #1\noindent \strut #2}
  \boxit{8pt}{\box4}}    

\newcommand{\todo}[1]{\vspace{5 mm}\par \noindent
\marginpar{\textsc{}} \framebox{\begin{minipage}[c]{0.95
\textwidth} \tt #1
\end{minipage}}\vspace{5 mm}\par}


%% taille du papier
\textwidth 17 true cm
\textheight 24 true cm
\addtolength{\hoffset}{-1.5cm}
\addtolength{\voffset}{-1.5cm}


%% D\'efinition des nouvelles commandes

\newcommand{\be}{\begin{equation}}
\newcommand{\ee}{\end{equation}}
\newcommand{\ba}{\begin{array}}
\newcommand{\ea}{\end{array}}
\newcommand{\bp}{\begin{proof}}
\newcommand{\ep}{\end{proof}}


\newcommand{\Bk}{\color{black}}
\newcommand{\Rd}{\color{red}}
\newcommand{\Bl}{\color{blue}}

\newtheorem{lem}{Lemme}[section]
\newtheorem{thm}[lem]{Th\'eor\`eme}
\newtheorem{prop}[lem]{Proposition}
\newtheorem{cor}[lem]{Corollaire}
\newtheorem{defin}[lem]{D\'efinition}
\newtheorem{exa}[lem]{Exemple}
\newtheorem{xca}[lem]{Exercice}
\newtheorem{remk}[lem]{Remarque}

\numberwithin{equation}{section}
\numberwithin{figure}{section}




\pagestyle{fancy}
\addtolength{\headheight}{\baselineskip}

\fancyhf{} 
\fancyhead[C]{\large{\bf\emph{}}
}

\fancyfoot[R]{\emph{UJF - IF - 2012--13}}
\fancyfoot[L]{\emph{GGMAT35b}}
\renewcommand{\headrulewidth}{0.2pt} %0.4
\renewcommand{\footrulewidth}{0.2pt} %0

\begin{document}

\emboite{\hsize}{Universit\'e Joseph Fourier
  \hfill \oldstylenums{2012}-\oldstylenums{2013}\\
  Master 1 \hfill Majeure Math\'ematiques
  \vskip7pt
  \centerline {\textsf{\textbf{Proposition de sujet de TER}}}
  \vskip7pt
 }

\bigskip


%---- Ensembles : entiers, reels, complexes... ----
\newcommand{\Nn}{\mathbb{N}}
\newcommand{\Zz}{\mathbb{Z}}
\newcommand{\Qq}{\mathbb{Q}}
\newcommand{\Rr}{\mathbb{R}}
\newcommand{\Cc}{\mathbb{C}}
\newcommand{\Kk}{\mathbb{K}}
\newcommand{\Hh}{\mathbb{H}}

\newcommand{\N}{\mathbb{N}}
\newcommand{\Z}{\mathbb{Z}}
\newcommand{\Q}{\mathbb{Q}}
\newcommand{\R}{\mathbb{R}}
\newcommand{\C}{\mathbb{C}}
\newcommand{\K}{\mathbb{K}}


\section{Titre}

\section{R\'esum\'e}

\section{Pr\'erequis}

\section{R\'ef\'erences}

%--------------------------
\end{document}