\documentclass[10pt,a4paper]{article}
\parindent=0cm

   \textheight=23,5cm
   \textwidth=15,7cm
\oddsidemargin=0cm

\usepackage[frenchb]{babel}
\usepackage[utf8]{inputenc}


\usepackage{url,amsmath,amssymb,amsthm,amsfonts,epsfig,eurosym,geometry}
\selectlanguage{french}

%pour écrire le code
\usepackage{listings}
\lstset{language=python, breaklines=true}


%pour faire des commentaires:
\usepackage{xargs}
\usepackage[prependcaption]{todonotes}
\newcommandx{\simon}[2][1=]{\todo[inline, author={Simon}, linecolor=blue,backgroundcolor=blue!25,bordercolor=blue,#1]{#2}}
\newcommandx{\simonnote}[2][1=]{\todo[author={Simon}, linecolor=blue,backgroundcolor=blue!25,bordercolor=blue,#1]{#2}}
%fin commentaires

\def\var{\mathrm{var}}
\def\BB{\mathcal{B}}
\def\FF{\mathcal{F}}
\def\Pa{\mathcal{P}}
\def\M{\mathcal{M}}
\def\P{\mathcal{P}}
\def\T{\mathcal{T}}
\def\L{\mathcal{L}}

\def\AA{\mathfrak{A}}
\def\S{\mathfrak{S}}
\def\SS{\mathcal{S}}
\def\CC{\mathcal{C}}

\def\A{\mathbb{A}}
\def\B{\mathbb{B}}
\def\D{\mathbb{D}}
\def\U{\mathbb{U}}


\def\bq{\begin{quote}\begin{em}}
\def\eq{\end{em}\end{quote}}

\def\un{\mathbf{1}}
\def\dd{\mathrm{d}}
\def\ee{\mathrm{e}}
\def\ii{\mathrm{i}}
\def\eps{\varepsilon}

\def\hth{\widehat{\theta}}
\def\ovx{\overline{X}}
\def\ima{\mathrm{Im}}

\newcommand{\C}{\mathbb C}
\newcommand{\R}{\mathbb R}
\newcommand{\N}{\mathbb N}
\newcommand{\Z}{\mathbb Z}
\newcommand{\Q}{\mathbb Q}

\newcommand{\norm}[1]{{\left\|{#1}\right\|}}
\newcommand{\ent}[1]{{\left[{#1}\right]}}
\newcommand{\abs}[1]{{\left|{#1}\right|}}
\newcommand{\scal}[1]{{\left\langle{#1}\right\rangle}}
\newcommand{\pare}[1]{{\left({#1}\right)}}

\newcommand{\espmes }{(\Omega, {\cal F})}
\newcommand{\espprob}{(\Omega, {\cal F}, P)}
\newcommand{\va}{variable al\'eatoire }
\newcommand{\vas}{variables al\'eatoires }
\newcommand{\Vas}{Variables al\'eatoires }
\newcommand{\cov}{\hbox{cov}}
\newtheorem{theo}{Théorème}[section]
\newtheorem{coro}{Corollaire}[section]
\newtheorem{lemm}{Lemme}[section]
\newtheorem{defi}[theo]{Définition}
\newtheorem{prop}[theo]{Proposition}
\newtheorem{rque}[theo]{Remarque}
\newtheorem{exem}[theo]{Exemple}
\newtheorem{exer}{Exercice}
\def\card{\mathrm{card}\,}



\newcommand{\tran}[1]{\,{}^{t}\!#1}
\renewcommand\det{\mathop{\textrm{d\'et}}}

\renewcommand{\le}{\leqslant}
\renewcommand{\ge}{\geqslant}
\def\demi{\frac{1}{2}}
\def\ds{\displaystyle}
\def\btheo{\begin{theo}}\def\etheo{\end{theo}}
\def\bcoro{\begin{coro}}\def\ecoro{\end{coro}}
\def\blemm{\begin{lemm}}\def\elemm{\end{lemm}}
\def\bdefi{\begin{defi}}\def\edefi{\end{defi}}
\def\bprop{\begin{prop}}\def\eprop{\end{prop}}
\def\brque{\begin{rque}}\def\erque{\end{rque}}
\def\bexem{\begin{exem}}\def\eexem{\end{exem}}

\def\bexer{\begin{exer}\begin{em}}\def\eexer{\end{em}\end{exer}}
\usepackage{graphics}
\usepackage{color}              % Need the color package
\usepackage{epsfig}
\def\F{{\cal F}}
\def\1{{\rm 1\kern-.8ex 1}}
\def\euro{\mbox{\raisebox{.25ex}{{\it =}}\hspace{-.5em}{\sf C}}}
\newcommand{\bin}[2]{\left( \begin{array}{c}
#1 \\#2\end{array}\right)}
\begin{document}

\noindent Intro. mod. num. \hfill Feuille 2 : Syst\`emes, conditionnement \hfill {L3 \the\year}\\

\bexer 
\begin{enumerate}
\item
Avec la méthode d'Euler, on calcule d'abord $y$ en faisant $L_2-\frac{1}{\eps}L_1$, on trouve
\[
y=\frac{2-\frac{1}{\eps}}{1-\frac{1}{\eps}}~.
\]
ce qui est proche de $1$ quand $\eps$ est petit. Avec une précision de $\eps$, en écrivant $m(x)$ la version "numérique" de $x$, on a (quand $\eps\rightarrow 0$):
\[
m\left(1-\frac{1}{\eps}\right)=2-\frac{1}{\eps}+\delta_1\left(1+\frac{1}{\eps}\right)=(1-\frac{1}{\eps})(1+\delta_1+O(\eps))
\]
avec $\abs{\delta_1}\leq \eps$. Comme l'inversion préserve l'erreur relative, on a 
\[
m\left(\frac{1}{1-\frac{1}{\eps}}\right)=\frac{1}{1-\frac{1}{\eps}}(1+\delta_2)
\]
où $\abs{\delta_2}=O(\eps)$. En calculant de même l'erreur sur le numérateur, on obtient
\[
m(y)=y(1+\delta_3)
\]
où $\delta_3=O(\eps)$. L'erreur relative sur $m(y)$ reste donc proche de $\eps$. Cependant, la deuxième étape de la méthode d'Euler, en subsituant $y$ dans la première ligne, donne
\[
x=\frac{1}{\eps}(1-y)=\frac{1}{1-\eps}=1+O(\eps)~.
\]
Mais $m(1-y)=1-y+\delta_4(1+\abs{y})=1-y+O(\eps)$ donc 
\[
m(x)=1+O(1)~.
\]
L'erreur sur $x$ est donc du même ordre que $x$, ce qui est très mauvais !
\item En intervertissant les deux équations, on obtient 
\[
y=\frac{1-2\eps}{1-\eps}
\]
et 
\[
x=2-y~.
\]
En supposant une précision de $\eps$, on a une erreur relative $O(\eps)$ dans le calcul de $y$, et donc une erreur absolue en $O(\eps)$ dans le calcul de $x$. Inverser les lignes permet donc d'avoir une erreur acceptable.
\end{enumerate}
\eexer





\bexer Soit
$$A=\left(\begin{array}{cc} 1&4\\
    -1&2 \end{array}\right)~~~~B=\left(\begin{array}{ccc} -2& 3&0\\
    1&0&-4\\ 2&0&5 \end{array}\right)~.$$
\begin{enumerate}
\item On fait une élimination par pivot de Gauss en retenant les opérations. Pour la matrice $A$, on fait $L_1\rightarrow L_1$ et $L_2\rightarrow L_1+L_2$, ce qui revient à multiplier par la matrice 
\[
X_1=\begin{pmatrix} 1&0 \\ 1&1\end{pmatrix}~.
\]
On a donc 
\[
X_1 A=\begin{pmatrix} 1&4\\0&6 \end{pmatrix}
\]
donc $A=LU$ avec 
\begin{align*}
L=X_1^{-1}&=\begin{pmatrix} 1 & 0 \\ -1 & 1 \end{pmatrix} & U=\begin{pmatrix} 1&4\\0&6 \end{pmatrix}
\end{align*}

Pour la matrice $B$, il faut deux opérations:
\begin{align*}
X_1 B &= \begin{pmatrix} -2&3&0 \\0 & \frac{3}{2} & -4 \\ 0 & 3 & 5 \end{pmatrix} & X_1&=\begin{pmatrix} 1&0&0 \\ \frac{1}{2} & 1 & 0 \\ 1&0&1 \end{pmatrix} & X_1^{-1}&=\begin{pmatrix} 1&0&0\\-\frac{1}{2}&1&0\\-1&0&1\end{pmatrix} \\
X_2X_1B&=\begin{pmatrix}-2&3&0\\0&\frac{3}{2}&-4\\0&0&13\end{pmatrix} & X_2&=\begin{pmatrix}1&0&0\\0&1&0\\0&-2&1\end{pmatrix} & X_2^{-1}&=\begin{pmatrix} 1&0&0\\0&1&0\\ 0&2&1\end{pmatrix}
\end{align*}
d'où on conclut que $B=LU$ avec $U=X_2X_1 B$ comme ci-dessus et 
\[
L=X_1^{-1}X_2^{-1} =\begin{pmatrix} 1 & 0 & 0\\ -\frac{1}{2} & 1 &0 \\ -1&2&1 \end{pmatrix}~.
\]
\item On a $\det(B)=\det(L)\det(U)=1\times(-2)\times \frac{3}{2}\times 8=-24$.
\item On calcule d'abord $v=Ux$, qui satisfait $Lv=(1,2,3)$ ce qui donne $v_1=1$, $-\frac{1}{2}v_1+v_2=2$ donc $v_2=\frac{5}{2}$ et $-v_1+2v_2+v_3=3$ donc $v_3=-1$. Puis on résoud l'équation $Ux=v$, ce qui donne $13u_3=-1$ (on commence par la dernière ligne), et $\frac{3}{2} u_2-4u_3=\frac{5}{2}$ donc $u_2=19/13$, et $-2u_1+3 u_3=1$ donc $u_3=22/13$.
\item On a $B^{-1}=U^{-1} L^{-1}=U^{-1}X_2X_1$ donc il suffit d'inverser $U$, ce qui peut effectivement se faire par la résolution de 3 systèmes linéaires (sinon en terminant le pivot de Gauss). 
\end{enumerate}
\eexer

\bexer
Soit $\alpha$ un param\`etre r\'eel.
On consid\`ere la matrice
$$
A(\alpha)=
\left(
\begin{array}{ccc}
 \alpha & 1 \\
1&2
\end{array}
\right)
.$$
\begin{enumerate}
\item Pour $\alpha\neq 0$ on obtient la factorisation $LU$ de $A$ en une étape, 
\[
A= \begin{pmatrix} 1 &0 \\ \frac{1}{\alpha} & 1 \end{pmatrix} \begin{pmatrix} \alpha & 1 \\ 0 & 2-\frac{1}{\alpha} \end{pmatrix}~.
\]
Pour $\alpha=0$ on ne peut avoir une factorisation $LU$, car $L_{1, 1}U_{1, 1}=0$ donc $L$ ou $U$ est de rang $<2$, alors que $A$ est de rang $2$ pour $\alpha\neq \frac{1}{2}$.
\item Si $\alpha=0$ Python inverse les lignes, obtenant une décomposition $PLU$ où $P$ est la matrice de permutation correspondant à l'inversion des lignes. Noter que si $\abs{\alpha}<1$ Python inverse aussi les lignes, afin d'avoir le pivot le plus grand possible (cf exo 1).
\end{enumerate}
\eexer






\bexer Pour calculer un déterminant naïvement, il faut $n!$ fois $n$ multiplications et $n!-1$ additions (sans compter le temps de calcul de $\epsilon(\sigma)$ et le temps utilisé pour construire toutes les permutations).

Pour calculer l'inverse par la formule de Cramer, il faut calculer le déterminant 
de la matrice, puis les $n^2$ déterminants des mineurs de taille $n-1$, soit 
$n\times n!+n^2\times(n-1)\times(n-1)!=n^2\times n!$ multiplications.

Avec la décomposition $LU$, à l'utilisation du $k$-ième pivot on a besoin de $(n-k)^2$ multiplications (ou divisions), soit de l'ordre de $2 n^3/6$ multiplications pour le calcul de $U$. Le calcul de $L$ comme le produit des inverses des matrices de transvection prend également de l'ordre de $2n^3/6$ multiplications.

Pour une matrice $100*100$, le calcul par la méthode de Cramer prend $10^{160}$ multiplications, donc $10^{150}$ secondes, alors que la méthode $LU$ demande de l'ordre de $10^6$ multiplications, soit $10^{-4}$ secondes. 
\eexer


\bexer Calculer le d\'eterminant de la matrice de Hilbert de taille 50 puis 100
de mani\`ere num\'erique et de mani\`ere exacte. Comparer les
r\'esultats et temps de calcul.\\
Pour quelques matrices $M$ al\'eatoires de taille 100, 200, 300, calculez le
d\'eterminant de $M$, comparez avec la borne de Hadamard de $M$
(produit des normes des vecteurs colonnes). Qu'observe-t-on~?
\eexer



\bexer {\bf D\'ecomposition de Cholesky.}\\
Soit $A$ une matrice hermitienne d\'efinie positive. Il existe une unique matrice triangulaire sup\'erieure $C$ qui a des \'el\'ements diagonaux strictement positifs telle que $A=C^*C$ (d\'ecomposition de Cholesky).\\
\begin{enumerate}
\item On peut écrire les élements de $C$ comme inconnues puis résoudre les $n(n+1)/2$ équations données par l'égalité $C^* C=A$ successivement. 

Une méthode astucieuse est d'utiliser la décomposition $LU$. En effet si $D$ est la diagonale de $C$ alors $C^*D^{-1}$ est une matrice triangulaire inférieure avec uniquement des $1$ sur la diagonale, et $A=(C^*D^{-1}) (DC)$. Ainsi, par unicité de la décomposition $LU$, on a $L=C^*D^{-1}$ et $U=DC$. Cela suggère l'algorithme suivant: 
\begin{enumerate}
\item calculer $U$ dans la décomposition $LU$: on trouve
$
U=\begin{pmatrix}
1&2&1\\
0&9&-3\\
0&0&1
\end{pmatrix}
$
\item Pour tout $i$, on divise la $i$-ième ligne de $U$ par $\sqrt{U_{i,i}}$. On obtient
\[
C=\begin{pmatrix}
1&2&1\\
0&3&-1\\
0&0&1
\end{pmatrix}
\]
\end{enumerate}

On peut vérifier que la matrice $\tilde{L}$ obtenue en multipliant chaque colonne de $L$ par $\sqrt{U_{i,i}}$ est bien égale à $C^*$. Ce fait n'est vrai que si $A$ est définie positive. Si $A$ n'est pas symétrique, on a plus $\tilde{L}=C^*$. Si $A$ est symétrique mais pas définie positive, les $U_{i,i}$ ne sont pas tous strictement positifs, et on ne peut donc pas prendre leurs racines. 
2) On résout les systèmes $C^*v=b$ puis $Cx=v$ successivement, chacunes de ces résolutions étant rapide puisque $C$ est triangulaire, en résolvant par le bas (en commençant par $v_n$ pour calculer $v$, puis par le haut (en commençant par $x_1$) pour calculer $x$. 
\item a) La matrice de l'exo 8 du TD 1 est 2-bande.
b) Deux méthodes de résolution: en utilisant l'algorithme (plus naturel) ou par l'absurde (plus astucieux, mais moins intuitif). 

{\bf Méthode par construction:} Vu l'algorithme décrit ci-dessus, on sait que $C=D^{-\frac{1}{2}} U$ où $A=LU$ est la décomposition LU de $A$ et $D$ est la diagonale de $U$. Il suffit donc de montrer que $U$ est $p$-bande. On sait que $U$ est obtenu par un pivot de Gauss descendant. Pour tout $n$ on note $A_n$ la matrice obtenue dans le pivot de Gauss après $n$ opérations. Montrons par récurrence que $A_n$ est $p$-bande pour tout $n$.

On a $A_0=A$ donc $A_0$ est $p$-bande.

De plus, soit $n\geq 0$ tel que $A_n$ soit $p$-bande. Notons $b_{i,j}$ le coefficient d'indice $i,j$ de $A_n$ et $c_{i,j}$ le coefficient d'indice $i,j$ de $A_{n+1}$. Pour calculer $A_{n+1}$, on replace une ligne $L_l$ par $L_l-\lambda L_k$ pour une ligne $L_k$ et un coefficient $\lambda$. Le coefficient utilisé dans le pivot est nécessairement $b_{k,k}$, donc $\lambda=\frac{a_{l,k}}{b_{k,k}}$. De plus on fait un pivot descendant, donc $k<l$. Enfin, $A_n$ est $p$-bande, donc $b_{l,i}=0$ pour $i\leq l-p$. Ainsi, on peut supposer que $k>l-p$ (sinon $\lambda=0$ et l'opération est triviale). De plus, puisque $b_{k,k}$ est utilisé comme pivot on a $b_{k,i}=0$ pour tout $i<k$, donc pour tout $i\leq l-p<k$ on a  
\[
c_{l,i}=b_{l,i}-\lambda b_{k,i}=0~.
\]
Il reste à montrer que $c_{l,i}=0$ pour tout $i\geq l+p$. Or, pour $i\geq l+p$, on a $b_{l,i}=0$ et $i\geq k+p$ donc $b_{k,i}=0$, d'où $c_{l,i}=0$. Donc $c_{l,i}=0$ dès que $\abs{l-i}\geq p$, et $c_{j,i}=b_{j,i}$ si $j\neq l$, donc $c_{j,i}=0$ dès que $\abs{l-i}\geq p$, et donc $A_n$ est $p$-bande. 


{\bf Méthode par l'absurde:} Supposons que $C$ ne soit pas $P$-bande, et soit $(i, j)$ avec $\abs{i-j}>p$ tel que $C_{i,j}\neq 0$ avec $i$ minimal (comme $C$ est triangulaire superieur on a alors $i<j$). Alors, en notant $[C^*C]_{i,j}$ le coefficient $i,j$ de $C^*C$, on a
\[
[C^*C]_{i,j}=\sum_{i=1}^n \overline{C}_{k, i} C_{k, j}
\]
mais $C$ est triangulaire supérieure, donc $C_{k,i}=0$ pour $k>i$, et comme $i$ est minimal tel que $C_{k,j}$ soit non nul, on a $C_{k,j}=0$ pour $k<i$. Donc 
\[
[C^*C]_{i,j}=\overline{C}_{i,i} C_{i,j}~.
\]
Comme $A$ est définie positive, $C$ est inversible, mais $det(C)=\prod_{k=1}^n C_{k,k}$ donc $C_{i,i}\neq 0$ et donc $[C^*C]_{i,j}\neq 0$, ce qui est absurde car $A=C^*C$ est $p$-bande.
\end{enumerate}

\eexer


\bexer 
\begin{enumerate}
\item De manière générale, on considère $A=\begin{pmatrix} a & b \\ b& a \end{pmatrix}$ avec $a, b\in \R$. Cette matrice est symétrique, donc elle admet une base orthonormale de vecteurs propres. On montre facilement qu'on peut prendre comme vecteurs propres $v_1=\frac{1}{\sqrt{2}}\begin{pmatrix} 1 \\ 1 \end{pmatrix}$ et $v_2=\frac{1}{\sqrt{2}}\begin{pmatrix} 1 \\ -1 \end{pmatrix}$ avec valeurs propres respectives $\lambda_1=a+b$ et $\lambda_2=a-b$.

\item Notons $Q_1$ le projecteur orthogonal sur l'espace propre $\R v_1$ et $Q_2=I-Q_1$ le projecteur orthogonal sur $\R v_2$. On a alors 
\[
A=\lambda_1 Q_1 +\lambda_2 Q_2
\]
et donc $A^{-1}=\lambda_1^{-1}Q_1+\lambda_2^{-1}Q_2$, d'où
\[
x=\lambda_1^{-1} Q_1 b+\lambda_2 Q_2 b~.
\]
\item En prenant $b=\begin{pmatrix} 1\\1 \end{pmatrix}=\sqrt{2}v_1$ on a $Q_1 b=b$ et $Q_2 b=0$, donc $x=\lambda_1^{-1} b=\frac{1}{2001} b$. 

Pour obtenir une grande erreur relative sur $x$, on choisit une erreur sur $b$ égale à $10^{-2} \sqrt{2} v_2$. On a bien une erreur relative sur $b$ petite:
\[
\frac{\norm{\Delta b}}{\norm{b}}=10^{-2}
\]
mais $Q_1\Delta b=0$ et $Q_2 \Delta b=\Delta b$ donc 
\[
x+\Delta x=\lambda_1^{-1} b+\lambda_2^{-1}\Delta b
\]
et donc $\Delta x =\lambda_2^{-2} \Delta b$. D'où
\[
\frac{\norm{\Delta x}}{\norm{x}}=\frac{\lambda_1}{\lambda_2}\simeq 20~.
\]
\end{enumerate}
\eexer

\vspace{.2cm}

\bexer 
\begin{enumerate}
\item On dévelloppe l'expression $(A+\Delta A) (x+\Delta x)=b+\Delta b$ et on élimine en utilisant $Ax=b$. On obtient 
\[
A \Delta x+\Delta A(x+\Delta x)=\Delta b
\]
d'où
\[
\Delta x=A^{-1} \left(\Delta b-\Delta A (x+\Delta x)\right)~.
\]
En utilisant la définition de la norme matricielle et l'inégalité triangulaire on en déduit que
\[
\norm{\Delta x} \leq ||| A^{-1} ||| \left(\norm{\Delta b}+|||\Delta A|||(\norm{x}+\norm{\Delta x}) \right)~.
\]
On a donc
\begin{align}\label{ineq1}
\left(1-|||A^{-1}|||~ |||\Delta A|||\right) \Delta x\leq ||| A^{-1} ||| \left(\norm{\Delta b}+|||\Delta A|||\norm{x}\right)~.
\end{align}
Par définition de $\kappa(A)$ on a 
\[
1-|||A^{-1}|||~ |||\Delta A|||=1-\kappa(A) \frac{|||\Delta A|||}{|||A|||}
\]
qui est $>0$ par hypothèse de l'énoncé. De plus $Ax=b$ donc $\frac{|||A||| \norm{x}}{\norm{b}}\leq 1$ et donc
\[
|||A^{-1}||| \norm{\Delta b}\leq \kappa(A)\frac{\norm{\Delta b}}{\norm{b}} \norm{x}~.
\]
En multipliant l'inégalité \eqref{ineq1} par $\frac{\norm{x}}{1-\kappa(A) \frac{|||\Delta A|||}{|||A|||}}$ on obtient le résultat demandé.
\item C'est une simple conséquence du fait que 
\[
||| A |||\leq \max \{ \abs{\lambda}~|~\lambda \text{ valeur propre de } A\}
\]
(avec égalité si $A$ est diagnonalisable), et du fait que les valeurs propres de l'inverse sont les inverses des valeurs propres. 

Dans l'exercice 7 on a une matrice diagonalisable, avec $|||A|||=\lambda_1=2001$ et $|||A^{-1}|||=\lambda_2^{-1}=1$, donc $\kappa(A)=2001$. Si on prend $\Delta A=0$ et $\norm{\Delta b}=10^{-2}$ dans l'inégalité de la question 8.1, on trouve
\[
\frac{\norm{\Delta x}}{\norm{x}}\leq 20.01~.
\]
On est donc dans le cas d'égalité pour $\Delta b=10^{-2} \sqrt{2} v_2$. 
\item Une matrice unitaire vérife $|||A|||=|||A^{-1}|||=1$ donc son conditionnement est $1$.
\end{enumerate}

\eexer



\end{document}
