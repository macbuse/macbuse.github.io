\documentclass[12pt,a4paper]{amsart}
\usepackage[utf8]{inputenc}
\usepackage{amsmath}
\usepackage{amsfonts}
\usepackage{amssymb}

\usepackage{hyperref}

\usepackage{float}
\usepackage{subfig}


%\usepackage[dvipdfmx]{graphicx}
\usepackage{graphicx}
\usepackage{caption}
%\usepackage[nobysame, alphabetic]{amsrefs}
%\usepackage{here}
%\usepackage{showkeys}
\newcommand{\modif}{$\clubsuit$}
\newtheorem{thm}{Theorem}[section]
\newtheorem{defn}[thm]{Definition}
\newtheorem{coro}[thm]{Corollary}
\newtheorem{prop}[thm]{Proposition}
\newtheorem{lem}[thm]{Lemma}
%\theoremstyle{definition}
\newtheorem{rmk}[thm]{Remark}
\newtheorem{cond}[thm]{Condition}

%CB defs
\def\HH{\mathbb{H}}
\def\dHH{\partial \mathbb{H}}
\def\HHn{\HH^n}
\def\hd{\hat{\delta}}
\def\ha{\hat{\alpha}}
\def\haa{\ha \cup \{\alpha^+,\alpha^-\}}

\def\im{\mathrm{Im}\,}
\def\oo{\HH / \Gamma_0(2)} 


\def\xx{\HH/g2}


\def\ZZ{\mathbb{Z}}
\def\CC{\mathbb{C}}
\def\RR{\mathbb{R}}
\def\QQ{\mathbb{Q}}
\def\NN{\mathbb{N}}

\def\tt{\Sigma_{1,1}}

\def\fp{\mathbb{F}_p}
\def\aut{\text{Aut}(\F2)}
\def\gl2{\mathrm{GL}(2, \ZZ)}
\def\sl2{\mathrm{SL}(2, \ZZ)}
\def\g2{\Gamma(2)}
\def\slc{\mathrm{SL}(2, \CC)}

\def\oi{\Gamma.\{i\}}

\def\xx{\HH/\g2}
\def\gg{\mathcal{G}_n}
\def\ggp{\mathcal{G}_p}

\def\isom{\mathrm{isom}(\HH)}

\def\isomH{\text{isom}^+(\HH)}
\def\tr{\text{tr\,}}


\def\GI{\mathbb{Z}[i]}
\def\hc{\CC \setminus \GI}



\title{Geodesics and values of quadratic forms}

 \author[McShane]{Greg McShane}
 % \author[Vlad]{Vlad Sergesciu}
\address{Institut Fourier 100 rue des maths, BP 74, 38402 St Martin d'H\`eres cedex, France}
\email{mcshane at univ-grenoble-alpes.fr}




\begin{document}

\maketitle

\section{Introduction}

	

\begin{thm}\label{triv}
Let $p$ be a prime then the equation
$$x^2 = -1$$
admits a solution in $\fp$ iff 
$p =2$ or $p-1$ is a multiple of $4$.
\end{thm}


\begin{thm}[Fermat]\label{main}
Let $p$ be a prime then the equation
$$x^2 + y^2 = p $$
has a solution in integers  iff  $p =2$ or $p-1$ is a multiple of $4$.
\end{thm}

There are many proofs of these theorems but the approach initiated
by Heath-Brown in \cite{heath} has inspired many admirers if not 
imitators see for example the  account of Elsholtz \cite{elsholtz}.
The essential ingredients are :
a finite set $X$ equipped with a pair of involutions
\begin{itemize}
	\item any fixed point of the one of the involutions, should it exist, is a solution of the equation.
	\item the other involution has a unique fixed point which is easy to compute.
\end{itemize}

The existence of the unique fixed point of the second involution
allows one to conclude that  the $X$ has an odd number of elements
and so that any involution has a fixed point.

The transformation $z \mapsto z + 1$ generates an infinite cyclic
group acting on $\mathbb{H}$.
The standard fundamental domain for this group is an infinite strip,
which we will refer to as the \textit{fundamental strip},
consisting of all the $z\in \CC$ such that 
the real part is between $0$ and $1$.

\begin{lem} \label{squares}
Let $n\geq2$ be an integer.
The number of  ways of writing $n$  as a  sum of squares
$$n = c^2 + d^2$$
with $c,d$ coprime integers
is equal to the number of points
of $\oi$, 
the $\sl2$  orbit of $i$,
in the fundamental strip at height $\frac{1}{n}$.
\end{lem}


\proof  Suppose there is such a point which we denote  $w$
verifying the hypotheses in particular

$$w = \frac{ai + b}{ci + d},\, \begin{pmatrix} a&b\\c&d
\end{pmatrix} \in \Gamma,$$
then

$$\im w = \im  \frac{ai +b}{ci+d } = \frac{\im i} {c^2 + d^2}.$$

Conversely if $c,d$ are coprime integers 
 then there exists $a,b$ such that
 $$ad - bc = 1 \Rightarrow  
 \begin{pmatrix}
 a & b \\
 c & d
 \end{pmatrix} \in \sl2.
$$
%The real part of $w = \frac{ai +b}{ci+d }$ is
%$$ac + bd = (a,b).(c,d),$$
%and
By applying a suitable iterate of the parabolic transformation 
$z \mapsto z + 1$,
one can choose $w$ such that $0 \leq \text{Re\,} w < 1$.
So if $n = c^2 + d^2$ then $\frac{ai +b}{ci+d }$
is on one of the lines of the family in the statement.

\hfill $\Box$




The principal congruence subgroup $\Gamma(p)$ is the subgroup of $\SL(2,\ZZ)$ is a normal.
It is a subgroup of $\Gamma_0(p)$:
$$ \begin{pmatrix} a & b \\ c & d \end{pmatrix} = \begin{pmatrix} 1 & * \\ 0 & 1 \end{pmatrix} \mod p.$$

For $p=2$ this is generated by just two elements namely:
$$ P = \begin{pmatrix} 1 & 1 \\ 0 & 1 \end{pmatrix} \text{ and } Q=  \begin{pmatrix} 1 & 0 \\ 2 & 1 \end{pmatrix}.$$
The product $P^{-1}Q$ is an element of order $2$:
$$ P^{-1}Q = \begin{pmatrix} -1 & -1 \\ 2 & 1 \end{pmatrix}.$$
So the quotient $\oo$ is a non-compact orbifold with two cusps and a single cone point.



\thebibliography{99}

\bibitem{aigner}
M. Aigner
\textit{Markov's Theorem and 100 Years of the Uniqueness Conjecture}, Springer( 2013)

\bibitem{aigner2}
Aigner M., Ziegler G.M.  
\textit{Representing numbers as sums of two squares.} In: Proofs from THE BOOK. Springer, Berlin, Heidelberg. (2010)

\bibitem{barag}
A. Baragar,
\textit{On the Unicity Conjecture for Markoff Numbers}
Canadian Mathematical Bulletin , Volume 39 , Issue 1 , 01 March 1996 , pp. 3 - 9

\bibitem{button}
J. O. Button, 
\textit{The uniqueness of the prime Markoff numbers},
 J. London Math. Soc.
(2) 58 (1998), 9–17.

% \bibitem{cana}
% Ilke Canakci, Ralf Schiffler
% \textit{Snake graphs and continued fractions}
% European Journal of Combinatorics
% Volume 86, May 2020, 103081

\bibitem{dolan}
Dolan, S. (2021). 105.38 A very simple proof of the two-squares theorem. The Mathematical Gazette, 105(564), 511-511. doi:10.1017/mag.2021.120

\bibitem{elsholtz}
Elsholtz C.A 
\textit{Combinatorial Approach to Sums of Two Squares and Related Problems.}
 In: Chudnovsky D., Chudnovsky G. (eds) Additive Number Theory. Springer, New York, NY.
 (2010) 


\bibitem{ford}
Lester R Ford,
\textit{Automorphic Functions}

\bibitem{heath}
Heath-Brown, Roger. 
\textit{ Fermat’s two squares theorem.} Invariant (1984) 

\bibitem{thesis}
G. McShane,
\textit{Simple geodesics and a series constant over Teichmuller space}
Invent. Math. (1998)

\bibitem{mong}
M.L. Lang, S.P Tan,
\textit{A simple proof of the Markoff conjecture for prime powers}
Geometriae Dedicata volume 129, pages15–22 (2007)

\bibitem{bob}
R. C. Penner, 
\textit{The decorated Teichmueller space of punctured surfaces}, 
Communications in Mathematical Physics 113 (1987), 299–339.


\bibitem{north}
Northshield, Sam. 
\textit{A Short Proof of Fermat’s Two-square Theorem.} The American Mathematical Monthly. 127. 638-638. (2020). 

\bibitem{serre}
J-P. Serre,
\textit{A Course in Arithmetic},
Graduate Texts in Mathematics,
Springer-Verlag New York
1973

% \bibitem{saw}
% Scott Wolpert,
% \textit{On the Kahler form of the moduli space of once-punctured tori}, 
% Comment. Math. Helv. 58(1983)246-256

\bibitem{zagier}
D. Zagier,
 \textit{A one-sentence proof that every prime p = 1 (mod 4) is a sum of two squares}, 
 American Mathematical Monthly, 97 (2): 144
 
 % \bibitem{zhang}
 % Y. Zhang,
 % \textit{ An elementary proof of uniqueness of Markoff numbers}
 % preprint, arXiv:math.NT/0606283
 
  % \bibitem{zhang2}
  %  Y. Zhang,
 % \textit{Congruence and uniqueness of certain Markoff numbers}
 % Acta Arithmetica, Volume: 128, Issue: 3, page 295-301






 


\end{document}


For the pair $\infty, 1/2$  and  $0, 2/p$   one has the matrix equation:
\begin{equation}
\begin{pmatrix}
1 & 1\\
0 & 2
\end{pmatrix}
= 
\begin{pmatrix}
 -(p-1)/2& 1\\
1 & 0
\end{pmatrix}
\begin{pmatrix}
0& 2\\
1 & p
\end{pmatrix}.
\end{equation}


The first matrix on the LHS is an element of $\gl2$
and from it, using Lemma \ref{fps}, we can obtain an orientation reversing isometry of
$\HH$ conjugating the inversion (\ref{inversion}) to $V$.
Finally, by changing signs in the expression(\ref{inversion})  one obtains an inversion 
conjugate to $V$ mapping the geodesic with endpoint $-1/p$ to itself.

%We say that a pair of (extended) rational $m/n, m'/n'$ are \textit{Farey neighbors} 
%if $|mn' - nm'| = 1$ with the convention that $\infty = 1/0$ 
%so that it's neighbors are exactly the integers $m/1$.
%The Farey neighbors of $0= 0/1$ are the rationals $1/n$ and in particular $1/p$ 
%is one of them though $2/p$ is not. 




