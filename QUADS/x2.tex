\documentclass[12pt,a4paper]{amsart}
\usepackage[utf8]{inputenc}
\usepackage{amsmath}
\usepackage{amsfonts}
\usepackage{amssymb}

\usepackage{hyperref}

\usepackage{float}
\usepackage{subfig}


%\usepackage[dvipdfmx]{graphicx}
\usepackage{graphicx}
\usepackage{caption}
%\usepackage[nobysame, alphabetic]{amsrefs}
%\usepackage{here}
%\usepackage{showkeys}
\newcommand{\modif}{$\clubsuit$}
\newtheorem{thm}{Theorem}[section]
\newtheorem{defn}[thm]{Definition}
\newtheorem{coro}[thm]{Corollary}
\newtheorem{prop}[thm]{Proposition}
\newtheorem{lem}[thm]{Lemma}
%\theoremstyle{definition}
\newtheorem{rmk}[thm]{Remark}
\newtheorem{cond}[thm]{Condition}

%CB defs
\def\HH{\mathbb{H}}
\def\dHH{\partial \mathbb{H}}
\def\HHn{\HH^n}
\def\hd{\hat{\delta}}
\def\ha{\hat{\alpha}}
\def\haa{\ha \cup \{\alpha^+,\alpha^-\}}

\def\im{\mathrm{Im}\,}
\def\oo{\HH / \Gamma_0(2)} 



\def\SL{\mathrm{SL}(2,\CC)}

\def\xx{\HH/g2}


\def\ZZ{\mathbb{Z}}
\def\CC{\mathbb{C}}
\def\RR{\mathbb{R}}
\def\QQ{\mathbb{Q}}
\def\NN{\mathbb{N}}

\def\tt{\Sigma_{1,1}}

\def\fp{\mathbb{F}_p}
\def\aut{\text{Aut}(\F2)}
\def\gl2{\mathrm{GL}(2, \ZZ)}
\def\sl2{\mathrm{SL}(2, \ZZ)}
\def\g2{\Gamma(2)}
\def\slc{\mathrm{SL}(2, \CC)}

\def\xx{\HH/\g2}
\def\gg{\mathcal{G}_n}
\def\ggp{\mathcal{G}_p}

\def\isom{\mathrm{isom}(\HH)}

\def\isomH{\text{isom}^+(\HH)}
\def\tr{\text{tr\,}}


\def\GI{\mathbb{Z}[i]}
\def\hc{\CC \setminus \GI}



\title{Geodesics and values of quadratic forms}

 \author[McShane]{Greg McShane}
 \author[Vlad]{Vlad Sergesciu}
\address{Institut Fourier 100 rue des maths, BP 74, 38402 St Martin d'H\`eres cedex, France}
\email{mcshane at univ-grenoble-alpes.fr}


\begin{document}

\maketitle



\begin{thm}\label{triv}
Let $p$ be a prime then the equation
$$x^2 = -1$$
admits a solution in $\fp$ iff 
$p =2$ or $p-1$ is a multiple of $4$.
\end{thm}


\begin{thm}[Fermat]\label{main}
Let $p$ be a prime then the equation
$$x^2 + y^2 = p $$
has a solution in integers  iff  $p =2$ or $p-1$ is a multiple of $4$.
\end{thm}

The principal congruence subgroup $\Gamma(p)$ is the subgroup of $\SL(2,\ZZ)$ is a normal.
It is a subgroup of $\Gamma_0(p)$:
$$ \begin{pmatrix} a & b \\ c & d \end{pmatrix} = \begin{pmatrix} 1 & * \\ 0 & 1 \end{pmatrix} \mod p.$$

For $p=2$ this is generated by just two elements namely:
$$ P = \begin{pmatrix} 1 & 1 \\ 0 & 1 \end{pmatrix} \text{ and } Q=  \begin{pmatrix} 1 & 0 \\ 2 & 1 \end{pmatrix}.$$
The product $P^{-1}Q$ is an element of order $2$:
$$ P^{-1}Q = \begin{pmatrix} -1 & -1 \\ 2 & 1 \end{pmatrix}.$$
So the quotient $\oo$ is a non-compact orbifold with two cusps and a single cone point.


\end{document}




