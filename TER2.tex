\documentclass[]{article}
\usepackage{lmodern}
\usepackage{amssymb,amsmath}
\usepackage{ifxetex,ifluatex}
\usepackage{fixltx2e} % provides \textsubscript
\ifnum 0\ifxetex 1\fi\ifluatex 1\fi=0 % if pdftex
  \usepackage[T1]{fontenc}
  \usepackage[utf8]{inputenc}
\else % if luatex or xelatex
  \ifxetex
    \usepackage{mathspec}
  \else
    \usepackage{fontspec}
  \fi
  \defaultfontfeatures{Ligatures=TeX,Scale=MatchLowercase}
\fi
% use upquote if available, for straight quotes in verbatim environments
\IfFileExists{upquote.sty}{\usepackage{upquote}}{}
% use microtype if available
\IfFileExists{microtype.sty}{%
\usepackage[]{microtype}
\UseMicrotypeSet[protrusion]{basicmath} % disable protrusion for tt fonts
}{}
\PassOptionsToPackage{hyphens}{url} % url is loaded by hyperref
\usepackage[unicode=true]{hyperref}
\hypersetup{
            pdfborder={0 0 0},
            breaklinks=true}
\urlstyle{same}  % don't use monospace font for urls
\IfFileExists{parskip.sty}{%
\usepackage{parskip}
}{% else
\setlength{\parindent}{0pt}
\setlength{\parskip}{6pt plus 2pt minus 1pt}
}
\setlength{\emergencystretch}{3em}  % prevent overfull lines
\providecommand{\tightlist}{%
  \setlength{\itemsep}{0pt}\setlength{\parskip}{0pt}}
\setcounter{secnumdepth}{0}
% Redefines (sub)paragraphs to behave more like sections
\ifx\paragraph\undefined\else
\let\oldparagraph\paragraph
\renewcommand{\paragraph}[1]{\oldparagraph{#1}\mbox{}}
\fi
\ifx\subparagraph\undefined\else
\let\oldsubparagraph\subparagraph
\renewcommand{\subparagraph}[1]{\oldsubparagraph{#1}\mbox{}}
\fi

% set default figure placement to htbp
\makeatletter
\def\fps@figure{htbp}
\makeatother


\date{}

\begin{document}

\subsection{Titre : l'Algorithme Zipper}\label{titre-lalgorithme-zipper}

\subsection{Résumé :}\label{ruxe9sumuxe9}

En analyse complexe, le théorème de l'application conforme, dû à
Bernhard Riemann, assure que toutes les parties ouvertes simplement
connexes du plan complexe qui ne sont ni vides ni égales au plan tout
entier sont conformes entre elles. Autrement dit, étant donné un domaine
homéomorphe au disque unité il existe une bijection holomorphe entre ce
disque et le disque unité.

Le théorème fut énoncé (sous l'hypothèse plus forte d'une frontière
formés d'arcs différentiables) par Bernhard Riemann dans sa thèse, en
1851. La démonstration de Riemann dépendait de l'existence d'une
solution du problème de Dirichlet sur le domaine, qui était considéré
comme àdmis à cette époque. Pourtant la méthode de Riemann ne s'applique
pas aux domaines simplement connexes de frontière non suffisamment lisse
; de tels domaines furent étudiés en 1900 par W. F. Osgood.

Au début des années 80, un algorithme élémentaire pour trouver une
bijection holomorphe a été découvert par R. Kühnau et D.E. Marshall.
L'algorithme est rapide et précis et relativement facile à programmer.
Soit \(z_0, ..., z_n\) des points distincts dans le plan complexe.
L'algorithme donne une suite de bijections holomorphes toute explicite
qui converge vers une bijection holomorphe entre le disque unité et une
région délimitée par une courbe de Jordan \(\gamma\) avec
\(z_0, ..., z_n \in \gamma\).

On va étudier la méthode utilisée pour établir la convergence dans
l'article de Rohde et Marshall. En particulier on va voir comment le
taux de convergence varie selon la regularité de la courbe \(\gamma\).

\subsection{Prérequis :}\label{pruxe9requis}

Fonctions holomorphes

\subsection{Références :}\label{ruxe9fuxe9rences}

Convergence of the Zipper algorithm for conformal mapping Donald E.
Marshall, Steffen Rohde https://arxiv.org/abs/math/0605532

\end{document}
