
\documentclass[11pt,a4paper]{article}
\usepackage[utf8]{inputenc}
\usepackage{geometry}
\usepackage{hyperref}
\geometry{margin=1in}

\begin{document}

\begin{center}
    {\Huge \textbf{Curriculum Vitae}} \\[0.5cm]
    {\Large Greg McShane} \\
    Institut Fourier, University of Grenoble Alps, France \\
    Email: \href{mailto:greg.mcshane@gmail.com}{greg.mcshane@gmail.com} \\
    Web: \url{https://macbuse.github.io/} \\
    GitHub: \url{https://github.com/macbuse} \\
\end{center}

\section*{Research Interests}
Low dimensional topology, Kleinian groups, moduli space of Riemann
surfaces, dynamics of the geodesic flow and interactions with
physics and number theory.

\section*{Research Grants and Leadership}
From 2015-2020 I was the director of the GDR des Tresses a research network in low
dimensional topology in France \url{http://tresses.math.cnrs.fr/}. The GDR is
a network of 700 researchers mainly in France but also in Italy, Spain, Canada
working on problems related to the braid group and low dimensional topology. We
organise a Winter School for young reserachers and two or three other conferences
per year.

From 2021-2025 I was  principal investigator for the ToFU project -
Topologie Effective et Calcul in the University of Grenoble
\url{https://persyval-lab.org/ToFu}. The goal of this project is to
combine the knowledge of theoretical mathematicians and computer
scientists in order to study questions in geometry which have a
combinatorial and algorithmic nature.


\section*{Academic Positions}
\begin{itemize}
    \item Professor, Institut Fourier, Grenoble (2008 -- present)
    \item Adjunct Professor, École Normale Supérieure Lyon (2014 -- present)
    \item Lecturer, Université Paul Sabatier, Toulouse (1997 -- 2008)
\end{itemize}

\section*{Education}
\begin{itemize}
    \item 2006: Habilitation diriger les recherches. Thesis: ``Identités pour l’espace des représentations du groupe fondamental d’une surface''
    \item 1992: Ph.D., Advisor: Prof. David B.A. Epstein. Thesis: ``A remarkable identity for lengths of curves on surfaces''
    \item 1988: M.Sc. (with distinction), University of Warwick. Thesis: ``Some notes on hyperbolic groups after Gromov''
    \item 1987: B.Sc. (with honours, class I), University of Glasgow (Glasgow, Scotland)
\end{itemize}

\section*{Selected Publications}
A full list of publications is available upon request.

\begin{itemize}

	\item Convexity and Aigner’s conjecture, to appear Advances in Math
	\item Isospectral configurations in Euclidean and Hyperbolic Geometry to appear Tohoku Journal of Math
	\item Geometry of Fermat’s sum of squares, to appear in Essays on Topology, Springer
	\item Rank two free groups and integer points on real  cubic surfaces,   arXiv:2009. 09829
	\item with H. Masai,  On systoles and ortho spectrum rigidity, Math Annalen, Vol 385, 2022  
	\item Geodesic intersections and isoxial Fuchsian groups. Annales de la faculté des sciences de Toulouse Mathématiques  Tome XXVIII, no 3 , p. 471-489 (2019)
	\item with William Goldman,  George Stantchev, Ser Peow Tan. Automorphisms of two-generator free groups and spaces of isometric actions on the hyperbolic plane,
	Memoirs of the AMS, (Book 259), 78 pages,  ISBN: 1470436140, (2019)
	\item with S. Kojima, Normalized entropy versus volume for pseudo-Anosovs, Geometry and Topology, 22(4) p. 2403-2426,    (2018)
\end{itemize}

\section*{Visiting Positions}
\begin{itemize}
    \item Fall 2019: Brown University, Providence, USA
    \item Spring 2018: Tsing Hua University, Beijing, China
    \item Spring 2015: University of Texas Austin, USA
    \item Winter 2015: CalTech, CA, USA
    \item Spring 2012: Tokyo Tech, Japan
    \item January 2008: NUS, Singapore
    \item January 2007: NUS, Singapore
    \item Fall 2005: IHES, Paris
    \item Fall 2004: Princeton University
    \item Spring 2004: UIC, Chicago
\end{itemize}

\section*{Invitations to Speak at Conferences}
\begin{itemize}
	\item Geometry and Dynamics in Low Dimensions" workshop in Hanoi, Vietnam, January 2025
	\item Special session on Markoff Triples at the
		upcoming Joint Mathematics Meetings in San
		Francisco, January 2024
	\item Thin Groups, NUS 	Singapore, June 2024
	\item Characters and Moduli of Surfaces, RIMS Kyoto, Japan, July 2023
	\item Geometry of discrete groups and hyperbolic
		spaces, Tokyo, Japan, June 2021
	\item Topology and Geometry of Low-dimensional Manifolds 5/6 – 8/6, 2019, Kanazawa, Japan.
	\item Workshop on Geometry of Moduli, Suzhou Chine 20/4-22/4/2018 Workshop on Geometry of Moduli, Suzhou Chine 20/4-22/4/2019.
	\item Geometric Structures and Representation Varieties , NUS Singapour, mai, 2017.
	\item Geometric Analysis, Metric Geometry and Topology, UGA, Grenoble, France juillet 2016
	\item Workshop on Teichmüller Theory and Low-Dimensional Topology, TSIMF China, janvier, 2016.
	\item 3-dimensional Geometric Structures, Representations of Surface Groups and related topics. Luxembourg 13 au 15 juillet, 2015.
	\item Dynamics and Geometry in the Teichmller Space, CIRM juillet, 2015. 
	\item Topology, Geometry and Algebra of Low-Dimensional Manifolds., Tokyo, Japan, mai 2015 .
	\item Identities in New York, CUNY Graduate Center, New York USA avril 2015 
	\item Mapping Class Groups and Teichmller Theory,  Nahsholim Israel mai, 2014. 
	\item Complex Analysis and Topology of Discrete Groups and Hyperbolic Spaces” RIMS, Kyoto, Japon janvier 2014.






\end{document}
