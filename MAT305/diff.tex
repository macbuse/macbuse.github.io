\documentclass{article}

\usepackage{amsmath}
\usepackage{amsfonts}
\usepackage{amssymb}


\begin{document}

\begin{center}
	{\Large MAT305 }
\end{center}

\hline
\vspace{.5in}

\textbf{Exercice 1:} Pour chaque application $f:\mathbb{R}^2
\rightarrow \mathbb{R}$ calculer sa dérivée $Df_x$ en chaque point.
\begin{enumerate}
	\item $f(x,y) = x+y$
	\item $f(x,) =  x-y$
	\item $f(x,y) = x^2+y^2$
	\item  $f(x,y) = x^2-y^2$
	\item $f(x,y) = cos(x)sin(y)$
\end{enumerate}

\hline
\vspace{.5in}


\textbf{Exercice 2 :} 
Pour chaque application $f:\mathbb{R}^3 \rightarrow \mathbb{R}^3$
calculer sa dérivée en chaque point.

\begin{enumerate}
	\item $f(x,y,z) = x+y+z$
	\item  $f(x,y,z) =  x-y+z.$
	\item $f(x,y,z) = x^2+y^2+z^2$
	\item $f(x,y,z) = x^2-y^2+z^2$
	\item $f(x,y,z) = \cos(x)\sin(y)$

\end{enumerate}

\vspace{1cm}
\hline
\vspace{.5in}

\textbf{Exercice 3:} Pour chaque application $f:\mathbb{R}^2
\rightarrow \mathbb{R}^2$
calculer sa dérivée en chaque point.
\begin{enumerate}
	\item $f(x,y) = (x+y, x-y)$
	\item $f(x,y) = (x-y, x+y)$
	\item $f(x,y) = (x^2+y^2, x^2-y^2)$
	\item $f(x,y) = (x^2-y^2, x^2+y^2)$
	\item $f(x,y) = (\cos(x)\sin(y), \cos(y)\sin(x))$
\end{enumerate}
\vspace{1in}
\end{document}
