\documentclass{article}

\usepackage{amsmath}


\begin{document}

\begin{center}
{\Large Diagonalisation bis}
\end{center}


\noindent
Faits :
\begin{enumerate}
	\item le produit des valeurs propres est \'egal au
		d\'eterminant de la matrices 
		$$\det A = \lambda_1 \lambda_2 \lambda_3$$
	\item la somme des valeurs propres est \'egale à la
		trace de la matrice 
		$$\mathrm{tr} A = a_{11}  + a_{22} + a_{33} = \lambda_1 + \lambda_2 + \lambda_3
		$$

\end{enumerate}


\noindent
Indication : chaque matrice admet $\lambda_1 = 1$ comme valeur propre : 
\begin{eqnarray}
 \det A&=&  \lambda_2 \lambda_3 \\ 
 \mathrm{tr} A &=& 1 + \lambda_2 + \lambda_3
\end{eqnarray}

On peut alors r\'esoudre ce syst\`eme 
pour trouver $\lambda_2$ et $\lambda_3$.

\vspace{1cm}
\hrule
\vspace{1cm}

\vspace{1cm}
\noindent
Calculer le deteminant, trouver les valeurs propres et vecteurs propres des matrices suivantes.


\begin{enumerate}

\item $$\begin{pmatrix}
0 & -1 & -1 \\ 
-2 & 7 & 2 \\ 
4 & -16 & -5
\end{pmatrix}$$

\item $$\begin{pmatrix}
7 & 0 & 2 \\ 
-7 & 2 & -3 \\ 
-12 & 0 & -3
\end{pmatrix}$$

\item $$\begin{pmatrix}
3 & -4 & 2 \\ 
4 & -7 & 4 \\ 
4 & -8 & 5
\end{pmatrix}$$

\item $$\begin{pmatrix}
3 & -2 & -4 \\ 
-12 & 5 & 12 \\ 
8 & -4 & -9
\end{pmatrix}$$

\item $$\begin{pmatrix}
0 & 1 & 0 \\ 
-2 & 3 & 0 \\ 
-4 & 2 & 2
\end{pmatrix}$$

\item $$\begin{pmatrix}
5 & 2 & 0 \\ 
-2 & 0 & -1 \\ 
4 & 2 & 1
\end{pmatrix}$$
\end{enumerate}
\end{document}
