\documentclass[a4paper,12pt]{article}

% \usepackage[french]{babel}


\usepackage[utf8]{inputenc}
\usepackage[T1]{fontenc}
\usepackage{amssymb,amsthm,amsmath,amsfonts}
\usepackage{textcomp}
%\usepackage{bbm}

\usepackage{framed}

\def\BF{\begin{framed} }
\def\EF{\end{framed} }

\frenchspacing

\long\def\/*#1*/{}

\def\gal{\text{Gal}}

\setlength{\textwidth}{15cm}
\setlength{\hoffset}{0.46cm}
\setlength{\oddsidemargin}{0cm}
\setlength{\marginparsep}{0cm}
\setlength{\marginparwidth}{0cm}
\setlength{\textheight}{24.2cm}
\setlength{\headheight}{0cm}
\setlength{\topmargin}{0cm}
\setlength{\headsep}{0cm}
\setlength{\paperwidth}{29cm}

\def\R{\mathbb{R}}
\def\C{\mathbb{C}}
\def\N{\mathbb{N}}
\def\Z{\mathbb{Z}}
\def\Q{\mathbb{Q}}
\def\F{\mathbb{F}}
\def\1{\mathbbm{1}}

% Usual sets of numbers  
\def\bA{{\Bbb A}}
\def\bC{{\Bbb C}}
\def\bF{{\Bbb F}}
\def\bK{{\Bbb K}}
\def\bN{{\Bbb N}}
\def\bP{{\Bbb P}}
\def\bQ{{\Bbb Q}}
\def\bR{{\Bbb R}}
\def\bZ{{\Bbb Z}}


\DeclareMathOperator{\pgcd}{pgcd}
\DeclareMathOperator{\ppcm}{ppcm}
\DeclareMathOperator{\GL}{GL}
\DeclareMathOperator{\Aut}{Aut}
\DeclareMathOperator{\Inn}{Inn}
\DeclareMathOperator{\id}{id}
\DeclareMathOperator{\End}{End}
\DeclareMathOperator{\Hom}{Hom}

\newcommand{\FF}{\mathbb{F}_q}

\def\ggauche{\textgravedbl}
\def\gdroite{\textacutedbl}


% Usual sets of numbers  
\def\bA{{\Bbb A}}
\def\bC{{\Bbb C}}
%\def{\bF{\Bbb F}}
\def\bK{{\Bbb K}}
\def\bN{{\Bbb N}}
\def\bP{{\Bbb P}}
\def\bQ{{\Bbb Q}}
\def\bR{{\Bbb R}}
\def\bZ{{\Bbb Z}}

\def\gc{\hbox{\goth c}}
\def\sevengc{\hbox{\sevengoth c}}
\def\mathfrak{\hbox{\goth S}}
\def\tr{\mathop{\rm tr}}
\def\Gal{\mathop{\rm Gal}}
\def\syquad#1#2{\left({#1\over #2}\right)}


\reversemarginpar

\newcommand{\dual}{\vee}

\newtheorem{enonce}{Exercice}
\newenvironment{E}[0]{\begin{enonce}\rm}{\bigskip \end{enonce}}

\begin{document}
	\noindent UGA \hfill L2\\
	\hfill 2023-24 \\
	\bigskip

	\begin{center}
		\textsc{CC1 MAT305}

% {\it Notes de cours autoris\'ees. Sujet de 2 pages.\\
% Il n’est pas necessaire de tout faire pour avoir la note maximale.}

{\bf You can write in english.}
		

	\end{center}
	\bigskip
	
\hrulefill
	
\vspace{1cm}

 

\begin{center}
{\large{\bf Exercice 1}}
\end{center}


Calculer les derivées partielles $f_x,f_y$ et $f_{xy}, f_{yx}$ pour
chacune des fonctions\\ 
$f:\mathbb{R}^2 \rightarrow \mathbb{R}$  suivantes :

 
 \begin{enumerate}
 
\item  $f(x,y) = x^3y^2+ 5y^2-x+7$

\item $ f(x,y) = \cos(xy^2)+\sin x$

\item $ f(x,y) = e^{x^2 + y^3}\sqrt{x^2+1}$

\item $f(x,y) = \frac{x^2 - y^2}{x^2 + 1}

 \end{enumerate}




\hrulefill

\vspace{1cm}

\begin{center}
{\large{\bf Exercice 2}}
\end{center}

Pour chacune des fonctions $f$ suivantes calculer $f_{xx} + f_{yy}$
: 

\begin{enumerate}
	\item $f(x,y)=  x^2 - y^2$
	\item $f(x,y) = e^x \sin(y) + e^y \cos(x)$
	\item $f(x,y)=  \log(\sqrt{ x^2 + y^2})$
	
\end{enumerate}

\hrulefill

\vspace{1cm}

\begin{center}
{\large{\bf Exercice 3}}
\end{center}


On définit $f:\mathbb{R}^2 \setminus (0,0)\rightarrow	\mathbb{R}$ par $f(x,y) = \log(x^2 +
y^2)$.

\begin{enumerate}
	\item Déterminer les ensembles de niveaux de $f$ et
		donner une interpretation geométrique.
	\item Monter que $f\circ \gamma$ est constante o\`u
		$\gamma: \mathbb{R} \rightarrow
		\mathbb{R}^2
		,\, t \mapsto (\cos(t),\sin(t))$ et préciser
		la valeur du constant.

	\item  Calculer le gradient de $f$ et $\dot{\gamma}$ le vecteur
		vitesse de $\gamma$. 
	\item Calculer le produit scalaire 
		$(\mathrm{grad}\,f).\dot{\gamma}$ et representer le résultat
		graphiquement.
	
\end{enumerate}
 

\end{document}
